\documentclass[a4paper,11pt]{kth-mag}
\usepackage[T1]{fontenc}
\usepackage{textcomp}
\usepackage{lmodern}
\usepackage[latin1]{inputenc}
\usepackage[swedish,english]{babel}
\usepackage{modifications}

\begin{document}
\removepagenumbers% Don't indent paragraphs.
\setlength\parindent{0em}
\setlength\parskip{1em}

\pagestyle{empty}

\selectlanguage{english}

\chapter*{Exjobb preliminary specification}

Adam Renberg, adamre@kth.se, \today

\begin{tabular}{lll}
	\textbf{Supervisor, Valtech:} & Erland Ranvinge & erland.ranvinge@valtech.se \\
    \textbf{Supervisor, KTH:}     & Narges Khakpour & khakpour@liacs.nl \\
\end{tabular}

\begin{tabular}{ll}
    \textbf{Website:} & http://tgwizard.github.com/thesis/ \\
    \textbf{Repository:} & http://github.com/tgwizard/thesis/ \\
\end{tabular}

The suggested topic is self-healing, and more specifically how to automatically detect "erroneous activities" when making changes to a system, in a "web site context". I will do the exjobb at Valtech.

The idea is to first survey the existing literature and see both what the current research is and what is done in the industry, by both bigger and smaller companies. This would result in the first part of the report.

The second part would be an implementation or modification of ideas from "some papers", industry ideas, or something similar. The implementation could hopefully be tested on "some project" at Valtech.

\paragraph{Things to do}

\begin{itemize}

\item The papers or ideas to make modifications/implementations of need to be found, and the ideas for the work need to be done drawn up.

\item Find project at Valtech where the as-of-yet unspecified and very vague implementation could be tested.

\item A time plan.

\item Finalize the specification, and get it approved by KTH.

\end{itemize}

It would be nice if the specification could be written and approved by \textbf{May 7}.

\end{document}
