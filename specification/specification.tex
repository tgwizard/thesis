\documentclass[a4paper,11pt]{kth-mag}
\usepackage[T1]{fontenc}
\usepackage{textcomp}
\usepackage{lmodern}
\usepackage[latin1]{inputenc}
\usepackage[swedish,english]{babel}
\usepackage{modifications}

\begin{document}
\removepagenumbers% Don't indent paragraphs.
\setlength\parindent{0em}
\setlength\parskip{1em}

\pagestyle{empty}

\selectlanguage{english}

\chapter*{Exjobb specification}

Suggested title: \textit{Test-driven runtime verification - An analysis
of the current research and a prototype for a new framework}.

\textbf{Adam Renberg}, 19880429-6211, 070-165 34 69, adamre@kth.se

\today

\begin{tabular}{lll}
	\textbf{Institute/School:} & School of Computer Science and Communication (CSC) at KTH \\
    \textbf{Company}           & Valtech \\
\end{tabular}

\begin{tabular}{lll}
	\textbf{Supervisor, Valtech:} & Erland Ranvinge & erland.ranvinge@valtech.se \\
    \textbf{Supervisor, KTH:}     & Narges Khakpour & khakpour@liacs.nl \\
\end{tabular}

\begin{tabular}{ll}
    \textbf{Website:} & http://tgwizard.github.com/thesis/ \\
    \textbf{Repository:} & http://github.com/tgwizard/thesis/ \\
\end{tabular}


The suggested topic is runtime verification (RV) of [web] systems. As suggested
by both Erland and Narges, this is of great interest in both the industry and
the academic community.

RV is a light-weight formal verification technique that is embedded in systems.
It verifies that certain properties hold at certain points in the code
\textit{during execution}. These properties are specified in a constructed
language, often a logic/calculus, and an RV framework instruments the system to
be monitored with code that verifies these properties. When a property
violation occurs, simple actions can be taken (e.g. log the error, send
emails), or more complex responses initiated, resulting in a self-healing
system.

The specification languages used by RV implementations is often based
on some formal logic and not written in the main language. In contrast,
offline testing frameworks often utilize the main language to great effect,
and their use is wide spread.

This exjobb will be an investigation into RV, how it is used today, and how an
RV framework could be implemented with inspiration from offline testing
frameworks. The idea is to find an RV specification syntax that closely
resembles that of offline testing frameworks, thus making RV fit better into
existing software development (processes), where offline testing is ubiquitous
and well accepted.

The exjobb will be split in these parts:

\begin{itemize}
	\item Background and state-of-the-art "inventory". What is RV? Why does it
        exist? Research results? How is it applied in practice? How is it used
        in web contexts? How are specification languages chosen and designed?
	\item Background and state-of-the-art of testing frameworks and their
        languages/syntax.
	\item Investigation of how RV can be applied to a [web] system. What
        specification language to use? In what system language: dynamic
        (python, ruby, etc.) static (Java, C\#)? How should code
        instrumentation be done?
	\item Implement an RV framework prototype. Find a suitable test-like
        specification language, or design it with inspiration from such
        languages.
	\item Apply implemented RV framework on a project at Valtech. Evaluate and
        analyze. Do some benchmarking to check on how the RV framework impacts
        performance.
\end{itemize}

Suggested project(s) at Valtech: The Valtech Intranet (python).

\textbf{Timeplan/Schedule}

I will work on the exjobb 50\% and 50\% on projects at Valtech, in periods of
two weeks exjobb, two weeks work. This fits well into the iteration-planning at
Valtech. During the summer I will work more on the exjobb, and use some of my
vacation hours for the exjobb.

\begin{itemize}
	\item Start: \textbf{April}.
    \item Specification done and approved: \textbf{May}.
	\item Final deadline: so that I can graduate in december 2012.
\end{itemize}

Suggested approximate time plan...

\begin{itemize}
    \item \textbf{One week} writing, and doing the research for, this
        specification.
	\item \textbf{Five weeks} of background and research.
	\item \textbf{One week} writing the background part of the report.
	\item \textbf{Four weeks} investigating and evaluating "what to do".
	\item \textbf{One week} writing about the investigation.
	\item \textbf{Three weeks} implementing and evaluating the RV framework.
	\item \textbf{Two weeks} testing and analysing the RV framework on "the
        project".
	\item \textbf{Three weeks} writing and finishing the report.
\end{itemize}

(Total: 20 weeks.)

Other required work related to the exjobb, such as doing the opposition of
another student's work, will take time from the work above and require
flexibility in planning.

\textbf{Issues}

\begin{itemize}
	\item Who will be the examiner?
    \item Do I need to start the exjobb work at a specific date and time? Do I
        need to/Should I take part in an exjobb "support group"?
    \item Do I need to finish my remaining courses before I start my exjobb?
        One exam remains, which I'll do on May 30th, and two reports.
	\item ...
\end{itemize}

\end{document}
