\documentclass[a4paper,11pt]{kth-mag}
\usepackage[T1]{fontenc}
\usepackage{textcomp}
\usepackage{lmodern}
\usepackage[latin1]{inputenc}
\usepackage[swedish,english]{babel}
\usepackage{modifications}
\usepackage{color}
\usepackage[usenames,dvipsnames]{xcolor}
\usepackage{colortbl}


\begin{document}
\removepagenumbers% Don't indent paragraphs.
\setlength\parindent{0em}
\setlength\parskip{1em}

\pagestyle{empty}

\selectlanguage{english}

\chapter*{Exjobb specification}

Suggested title: \textit{Test-oriented runtime verification - Using a test-like specification syntax for runtime verification}.

\textbf{Adam Renberg}, 19880429-6211, 070-165 34 69, adamre@kth.se

\today

\begin{tabular}{lll}
	\textbf{Institute/School:} & School of Computer Science and Communication (CSC) at KTH \\
    \textbf{Company}           & Valtech \\
\end{tabular}

\begin{tabular}{lll}
	\textbf{Supervisor, Valtech:} & Erland Ranvinge & erland.ranvinge@valtech.se \\
    \textbf{Supervisor, KTH:}     & Narges Khakpour & khakpour@liacs.nl \\
    \textbf{Examiner, KTH:}			  & ???	& ??? \\
    \textbf{Coordinator, KTH:}	  & Ann Bengtsson & ann@csc.kth.se \\
\end{tabular}

\begin{tabular}{ll}
    \textbf{Website:} & http://tgwizard.github.com/thesis/ \\
    \textbf{Repository:} & http://github.com/tgwizard/thesis/ \\
\end{tabular}


The suggested topic is runtime verification (RV) of [web] systems. As suggested
by both Erland and Narges, this is of great interest in both the industry and
the academic community. The exjobb will be done at Valtech.

RV is a light-weight formal verification technique that is embedded in systems.
It verifies that certain properties hold at certain points in the code
\textit{during execution}. These properties are specified in a constructed
language, often a logic/calculus, and an RV framework instruments the system to
be monitored with code that verifies these properties. When a property
violation occurs, simple actions can be taken (e.g. log the error, send
emails), or more complex responses initiated, resulting in a self-healing
system.

The specification languages used by RV implementations is often based
on some formal logic and not written in the main language. In contrast,
offline unit testing frameworks, such as \textit{JUnit} for Java or \textit{unittest} for python,
often utilize the main language to great effect,and their use is wide spread.

This exjobb will be an investigation into RV, how it is used today, and how an
RV framework could be implemented with inspiration from offline unit testing
frameworks. The idea is to find an RV specification syntax that closely
resembles that of offline unit testing frameworks, thus making RV fit better into
existing software development (processes), where unit testing is ubiquitous
and well accepted.

An important part of RV specifications is that they should be formal, and thus
that the properties they specify yield a formal proof of correctness for the
current execution. With a more lenient syntax for the specification language, such as
a fully fledged programming language, we need to ensure that the formal properties
still hold.

We need to:

\begin{enumerate}
	\item Specify the syntax for the RV specifications
	\item and relate these to a formal logic
	\item and transform them so that they can be executed in the running system
		(code instrumentation)
\end{enumerate}

We will focus on item 1 and 3.

The exjobb will be split in these parts:

\begin{itemize}
	\item Background and state-of-the-art "inventory". What is RV? Why does it
        exist? Research results? How is it applied in practice? How is it used
        in web contexts? How are specification languages chosen and designed?
	\item Background and state-of-the-art of testing frameworks and their
        languages/syntax.
	\item Investigation of how RV can be applied to a (web) system. What
        specification language to use? In what system language: dynamic
        (python, ruby, etc.) static (Java, C\#)? How should code
        instrumentation be done?
	\item Implement an RV framework prototype. Find a suitable test-like
        specification language, or design it with inspiration from such
        languages.
	\item \textit{(Possible extension)}
		Apply implemented RV framework on a project at Valtech. Evaluate and
        analyze. Do some benchmarking to check on how the RV framework impacts
        performance.
\end{itemize}

Suggested project(s) at Valtech: The Valtech Intranet (python).

\textbf{Timeplan/Schedule}

I will work on the exjobb 50\% and 50\% on projects at Valtech, in periods of
two weeks exjobb, two weeks work. This fits well into the iteration-planning at
Valtech. During the summer I will almost exclusively work on the exjobb, and also
use some of my vacation hours for the exjobb.

Suggested approximate time plan:

\begin{center}
	\renewcommand{\arraystretch}{1.5}
	\arrayrulecolor{Gray}
	\begin{tabular}{ l l l | p{7cm} }
		\textbf{Start} & \textbf{Weeks} & \textbf{N} & \textbf{Work} \\
		\noalign{\smallskip}\hline\noalign{\smallskip}
		
		May & \textit{1w in May} & \textbf{1w} &
		Writing, and doing the research for, this specification.
		\\
		
		May & \textit{2w in May}, v23, v26, v27 & \textbf{5w} &
		Doing background \& research.
		\\
		
		9/7 & v28 & \textbf{1w} &
		Writing the background part of the report.
		\\
		
		16/7 & v29, v30, v31, v32 & \textbf{4w} &
		Investigating and evaluating "what to do".
		\\
		
		13/8 & v33& \textbf{1w} &
		Writing about the investigation.
		\\
		
		20/8 & v34, v35, v36 & \textbf{3w} &
		Implementing and evaluating the RV framework.
		\\
		
		10/9 & v37, v38 & \textbf{2w} &
		Testing and analysing the RV framework on "the project".
		\\
		
		24/9 & v39, v40, v41 & \textbf{3w} &
		Writing and finishing the report.
		\\
		
	\end{tabular}
\end{center}

Total: 20 weeks. End date: \textbf{12 October 2012} (very optimistic).

I will weeks v24 and v25 off, as well as about 1 week in august (not planned). \textbf{The weeks I will be working at Valtech haven't been written into the time table.}

Other required work related to the exjobb, such as doing the opposition of
another student's work, will take time from the work above and require
flexibility in planning.

\textbf{Issues}

\begin{itemize}
	\item Who will be the examiner?
    \item Do I need to start the exjobb work at a specific date and time? Do I
        need to/Should I take part in an exjobb "support group"?
\end{itemize}

\end{document}
