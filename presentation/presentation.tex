% :autocmd BufWritePost * !pdflatex presentation.tex
\documentclass{beamer}
\usepackage[utf8]{inputenc}

\usetheme{Szeged}

\title{Test-inspired runtime verification}
\subtitle{Using a unit test-like specification syntax for runtime verification}
\author{Adam Renberg}
\institute{The School of Computer Science and Communication (CSC),\\KTH}
\date{March 17, 2013}

\begin{document}

\begin{frame}
  \titlepage
\end{frame}

\begin{frame}{TODO}
  \begin{itemize}
    \item Change style...

  \end{itemize}
\end{frame}

\section{Preface}

\begin{frame}{Introduction}
  This is an exjobb. I did it at Valtech. My supervisors have been Narges
  Khakpour (KTH) and Erland Ranvinge (Valtech). My examiner is Johan Håstad.
\end{frame}

\begin{frame}{Outline}
  \tableofcontents
\end{frame}


\section{Introduction}

\subsection{General Background}
\begin{frame}{General Background}
  General background so that I can introduce the problem statement.
\end{frame}

\subsection{Problem Statement}
\begin{frame}{Problem Statement}
  How can runtime verification specifications be written in a manner that uses
  the syntax of the target program's programming language, and resembles the
  structure of unit tests? Can we still give a formal semantics to the
  specification language, or a part of it? How can we bridge the spectrum of
  different approaches for verification, creating something in between the formal
  and informal techniques?
\end{frame}


\section{Background}

\subsection{Verification}
\begin{frame}{Background to Verification}
\end{frame}

\subsection{Runtime Verification}
\begin{frame}{What is Runtime Verification?}
\end{frame}

\subsection{Unit Testing}
\begin{frame}{What is Unit Testing?}
\end{frame}

\section{pythonrv}
\subsection{Syntax}
\subsection{Instrumentation}

\section{f-pythonrv}
\begin{frame}{A}
\end{frame}

\section{Evaluation}
\begin{frame}{Evaluation}
  Using \textit{pythonrv} on a real application: An intranet at Valtech.
\end{frame}

\section{Conclusions}
\begin{frame}{Conclusions}
  List the questions from the problem statement again, but now with answers.
\end{frame}

\begin{frame}{Future Work}
  List the future work section of the report.
\end{frame}

\begin{frame}{Final Thoughts}
  Program size and complexity is increasing. Verification, formal or informal,
  can help with software development. Runtime verification is nice. Make it
  simpler to use, and readily available, and people will use it.
\end{frame}
\end{document}
