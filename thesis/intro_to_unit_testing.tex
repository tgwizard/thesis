%================================================
%====== Chapter 3, Introduction to Unit Testing
%================================================

\pagestyle{newchap}
\chapter{Introduction to Unit Testing} \label{chapter-intro-to-unit-testing}

An approach to verification quite different from formal methods is
\emph{testing}, which is the practical approach of checking that the program,
given a certain input, produces the correct/acceptable output. Testing consists
of a set, or \textit{suite}, of \textit{test cases}. A test case tests some
aspect of the program, from large ("It should be possible to go through the
shop flow and buy a product") to small ("This function should throw an
exception when given a null argument").

Testing can be either manual or \textit{automated}, or something in between. In
manual testing a programmer, tester or other stakeholder, runs the program,
entering some inputs and inspecting the outputs. Test cases could be documents
describing what should happen for given inputs, where to click or type to cause
the desired outcome. In automated testing the test cases are
machine-runnable, so that a computer can run them during the development
process, or when checking code into a central repository, or during releases,
etc.

Testing is not complete (for all but the most trivial programs, it is
impossible to write complete tests), and lacks a formal foundation, so it
cannot be used for formal verification. Testing is popular and is used on
virtually all development projects, taking up a large part of development time
(see e.g.\ \cite{brooks75mythicalmanmonth}).


\section{Unit Testing}

\textit{Unit testing} is the concept of writing tests for small units in a
program, such as functions, classes, etc. The separate parts --- silos, with
minimum dependencies to the rest of the system --- are tested in isolation
during development to test their functionality. The aim is to reduce the risk
of breaking existing functionality when developing new features, or modifying
existing code, by preventing regression. Dividing a program into units can be
difficult, and proponents of unit testing argue that it encourages a decoupled
and modular system design.

Unit testing is quite young, perhaps having begun in earnest in the 90s, and it
was popularized by the extreme programming (XP)
movement\footnote{\texttt{http://www.extremeprogramming.org/}}. Testing in
general is very old.

When discussing testing, and unit testing in particular, we must mention the
concept of test-driven development (TDD). Also made popular by XP, it consists
of the cycle:
\begin{inparaenum}[1\upshape)]
  \item write a failing test;
  \item make it pass by writing the simplest code you can; and
  \item refactor --- rewrite the code, cleaning it up and giving it a better
    structure.
\end{inparaenum}
Tests here play the part of specifications for the units of the program.


\section{xUnit} \label{section-xunit}

Kent Beck introduced the style of the modern unit testing framework in his work
on a testing framework for Smalltalk \cite{becksmalltalktesting}. Together
with Eric Gamma he later ported it to Java, resulting in
\textit{JUnit}\footnote{\texttt{http://www.junit.org/}}.
Today, this has lead to \textit{frameworks} in several programming languages,
and they are collectively called xUnit \cite{fowlerxunit}. Examples include,
beside JUnit for Java:
\textit{unittest}\footnote{\texttt{http://docs.python.org/library/unittest.html}}
for Python,
\textit{NUnit}\footnote{\texttt{http://www.nunit.org/}} and
\textit{xUnit.net}\footnote{\texttt{http://xunit.codeplex.com/}} for C\#
and~.Net,
\textit{RSpec}\footnote{\texttt{http://rspec.info/}} and the
\textit{Test::Unit}\footnote{\texttt{http://www.ruby-doc.org/stdlib-1.9.3/libdoc/test/unit/rdoc/Test/Unit.html}}
package for Ruby,
and \textit{Jasmine}\footnote{\texttt{http://pivotal.github.io/jasmine/}} for
JavaScript.

The xUnit style of unit testing \cite{fowlerxunit} has given rise to unit
testing frameworks for many programming languages. Their structure are all
based on the same concept, and since JUnit is the canonical implementation, and
one of the first, we will use it for a short demonstration,
shown in Figure~\ref{figure-junit}. Jasmine, being younger and more ``hip'',
has a slightly different syntax, but essentially the same structure; see
Figure~\ref{figure-jasmine}.

\begin{figure}[h!]
	\begin{center}
	\begin{minipage}{0.9\textwidth}
		\lstset{language=Java}
		\lstinputlisting{figures/junit_example.java}
	\end{minipage}
	\end{center}
  \caption{An example of unit testing syntax, written Java as a test case for
    JUnit.}
	\label{figure-junit}
\end{figure}

\begin{figure}[h!]
	\begin{center}
	\begin{minipage}{0.9\textwidth}
		\lstset{language=JavaScript}
		\lstinputlisting{figures/jasmine_example.js}
	\end{minipage}
	\end{center}
  \caption{An example of unit testing syntax, written in JavaScript and
    Jasmine.}
	\label{figure-jasmine}
\end{figure}

In JUnit, and xUnit, the major constructs are the \textit{test suite},
comprising \textit{test cases}, which contain tests. In the example in
Figure~\ref{figure-junit}, the test suite is implicitly created by JUnit,
comprising the methods marked with the \texttt{@Test} annotation. The
\texttt{assert} keyword and functions like \texttt{assertEquals} are used to
verify the correctness of variables, program state, function outputs, etc. A
\textit{test runner} runs the test suite, reporting progress to the user. When
the tests are finished, any errors are displayed.

The example in Figure~\ref{figure-junit} has two tests, and methods to set
up and tear down the tests \textit{fixture}. These functions are usually called
\textit{setUp} and \textit{tearDown}, respectively, and are called before and
after each test. The fixture is the surrounding set of objects (environment)
that the object under test requires to work properly. Test written in the style
of Figure~\ref{figure-junit} are traditional unit tests.

Similar to the JUnit structure, Jasmine (Figure~\ref{figure-jasmine}) test
suites are created with the \texttt{describe} function, and test cases (called
specifications here) with the \texttt{it} function. The
\texttt{expect(\dots).toEqual(\dots)} style is used in favour of
\texttt{assert} for verification.

\section{Mocking and Faking} \label{section-mocking}

A common issue when writing unit tests is that, to instantiate an object X,
say, or to call a function Y, the program needs access to some other objects Z.
Z might be something simple, which we can easily create in the test. It might
also be a network or database connection, or something doing heavy calculation,
or just something complex.

One way to work around this is to create fake objects (also called mock-, fake-
or spy objects). A fake network connection has the same interface as a real
network connection, but calling it does not actually transmit anything
anywhere, and it might return pre-defined, hard coded data. Fake objects could
save what actions are taken upon them, and the test could then verify that
these are according to expectations.


\section{Expectations} \label{section-expectations}

Instead of writing fake objects, we can create a mock object and pre-record
what actions we expect to be taken upon them. This is called writing
\textit{expectations} \cite{fowler07expectations}. A simple example of
expectations is shown in Figure~\ref{figure-expectations}.

\begin{figure}[h!]
	\begin{center}
	\begin{minipage}{0.9\textwidth}
		\lstset{language=Java}
		\lstinputlisting{figures/expectations_example.java}
	\end{minipage}
	\end{center}

	\caption{An example of expectations, written using jMock and JUnit.
	Example taken from \cite{fowler07expectations}.}
	\label{figure-expectations}
\end{figure}

Figure~\ref{figure-expectations} shows a test of a fictional shop. The test
tests only one thing, the fill method of the Order object, but it requires a
Warehouse object, for access to the inventory. We supply a mock Warehouse, with
expectations on which methods should be called on it, with which arguments and
what they should return.

An expectation follows a simple pattern:

\begin{itemize}
	\item A function, with an optional object, which is expected to be called.
	\item An invocation count of how often the function is expected to be called.
	\item Expected arguments for the function call. These can be explicit values,
		or generic types, or rules defining the acceptable values.
	\item The return value and modifications to the global state; what should
		happen when the function is called.
	\item When the function call should happen, e.g.\ in what sequence of
		function calls, in what global state.
\end{itemize}

There are two common ways of specifying expectations: recording and explicit
specification. Figure~\ref{figure-expectations} shows an example of how to
explicitly specify expectations.

When recording expectations, the test setup creates a mock object and calls the
expected functions, with expected arguments and return values, in the expected
order. The mock is set into replay mode, and it will replay the recorded
expectations, and verify that they occur correctly.

There are several frameworks for working with expectations, such as
jMock\footnote{\texttt{http://www.jmock.org/}} for Java, Rhino
Mocks\footnote{\texttt{http://ayende.com/wiki/Rhino+Mocks.ashx}} for~.Net and
Ludibrio\footnote{\texttt{https://github.com/nsigustavo/ludibrio/}} for Python.
With both runtime verification and unit testing introduced, a combination of
the two will be the main subject in the next two chapters.

