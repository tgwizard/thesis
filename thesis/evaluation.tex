%================================================
%====== Chapter 5, Evaluation
%================================================

\pagestyle{newchap}
\chapter{Evaluation} \label{chapter-evaluation}

To see how \textit{pythonrv} would work in a real-world setting it was
incorporated into a real-time web application for Valtech Sweden, a
medium-sized Swedish company.

The web application is written in Python 2.7 using the
Django\footnote{\texttt{https://www.djangoproject.com/}} web framework. It has
approximately 10000 lines of code.

There are two questions we need to answer when writing specifications for a
program. First, when, in the life-cycle of the program, should we attach the
specifications? In other words, when should the code instrumentation be done?
And second, and most important: what specifications should be written, and for
which functions?

We can answer the first question first. It requires a bit of knowledge on the
start up sequence for, and structure of, Django applications.


\section{Technical Perspective}


\subsection{Anatomy of a Django Application}

A Django application follows the Model-View-Controller pattern, or as they call
it, the Model-Template-View pattern. The model is a representation of the data
used by the program, and the templates are the layer that constructs the
display for the user. The view links the two together, fetching the correct
models for specific requests, and then delegating to the appropriate templates.

Application-specific configuration for Django programs are stored in settings
modules, which are ordinary Python files. These contain settings for database
connections, authentication, etc. During startup, Django reads the settings
files, starts up its internal machinery, and waits for the first request.


\subsection{When to Attach}

At first glance it might seem desirable to attach the specifications before
even starting the Django framework. That way we could monitor the startup
process, and all of the functionality of Django.

A problem with this, that is due to how Python works, and how \textit{pythonrv}
does code instrumentation, is that \textit{pythonrv} needs to load the modules
(files) for each function to be monitored. These modules are often heavily
dependent on Django, and that it has been started correctly, with all settings
loaded.

A suitable time to instrument the program --- to enable the specifications ---
is during startup, after the settings have been loaded. Some specifications,
which do not monitor code dependent on the settings, could be loaded before
that.


\subsection{Technical Issues}

Early in the process of using \textit{pythonrv} in the web application it was
discovered that the copying of data, such as function arguments, that
\textit{pythonrv} does would not work with Django. The latest version of
Django, v1.4.1, uses a module called \texttt{cStringIO}, which produces objects
that cannot be copied. All functions dealing with web requests are affected by
this. This has been fixed in the development branch of Django, but in the
meantime, \textit{pythonrv} has an option to disable argument copying, either
for all specifications or for a subset of them, to work around this issue.


\section{Potential Value}

Now to the most important question: what specifications could, and should, be
written? What value do they provide?

The web application is only intended for internal use, for employees.
Authentication of employees and authorization of their access rights are very
important. Specifications can be written to make sure all views or requests are
made by an authenticated user (except for the login screen), and that they are
allowed to perform the requested action. A specification showing a
specification verifying that proper authentication has been done is shown in
Figure~\ref{figure-app-authentication-informal}. This is just one way of
writing a specification function for this requirement. There are several other
options, such as utilizing the temporal capabilities of runtime verification
with the \texttt{event.next} function. E.g., if a request asks for a resource
requiring authentication, we could specify that before the response is sent,
authentication must have been done.

\begin{figure}[h!]
	\begin{center}
	\begin{minipage}{0.9\textwidth}
	\lstinputlisting{figures/app_authentication_informal.py}
	\end{minipage}
	\end{center}

  \caption{An informal \textit{pythonrv} specification verifying that views are
    only accessible with proper authentication. Note how the use of Python in
    the specification function allows us to refactor away the condition that
    determines whether authentication should be required, leaving the
    specification much simpler and cleaner.}
	\label{figure-app-authentication-informal}
\end{figure}

\todo{more examples}
Talk about where specifications would be suitable, and for what.

