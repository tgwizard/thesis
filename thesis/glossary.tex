%\pagestyle{newchap}
\chapter{Glossary} \label{appendix-glossary}

This appendix consists of a small glossary with short explanations of a few
key concepts used in this report, and pointers to where they are described in
more detail.

\begin{description}
  \item[incompleteness] As in the incompleteness of testing, reflects the fact
    that, with testing, not all possible states are examined. Some formal
    methods are complete.

  \item[formal methods] Techniques for mathematical reasoning about program
    correctness. See Section~\ref{section-formal-methods}.

  \item[linear temporal logic, LTL] See Section~\ref{section-ltl}.

  \item[mock, mocking] Using fake, stand-in objects to isolate units of the
    program from the rest of the system. See Section~\ref{section-mocking}.

  \item[model, system model] A conceptual model and abstraction of a system.
    See Section~\ref{section-system-model}.

  \item[model checking] See Section~\ref{section-formal-methods}.

  \item[runtime verification] Verifying an execution of a program. See
    Section~\ref{section-rv} for an overview and
    Chapter~\ref{chapter-intro-to-rv} for a more in-depth description.

  \item[self-healing, self-adapting] The concept of systems that can analyse
    itself, react to events, and enact modifications to fix or circumvent
    errors, and possibly add improvements to existing functionality. See e.g.\
    \cite{huebscher08survey} for more.

  \item[specification] Something that describes the correct behaviour of
    something else. See Section~\ref{section-specifications}.

  \item[state explosion problem] Concerns the problem of the exponential
    increase in the state-space when more variables of a system are taken into
    consideration in the system model.

  \item[testing] An approach for program verification. See
    Chapter~\ref{chapter-intro-to-unit-testing}.

  \item[undecidability] A (decision) problem is undecidable if it is impossible
    to construct an algorithm for it that always gives the correct answer. One
    example of an undecidable problem is the \textit{halting problem}.

  \item[unit testing] Dividing the program into small units, testing each
    separately. See Chapter~\ref{chapter-intro-to-unit-testing}.

  \item[verification] Checking the correctness of a program, by using
    techniques such as testing or formal methods. See
    Section~\ref{section-definition-verification} for a more thorough
    definition.
\end{description}

