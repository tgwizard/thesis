%================================================
%====== Chapter 1, Introduction
%================================================

\pagestyle{newchap}
\chapter{Introduction} \label{chapter-introduction}

Due to the increasing size and complexity of computer software it has become
increasingly difficult, if not impossible, to assure that the software
works as desired. This is where verification can be helpful. Of the various
approaches used for verification, \textit{testing}, in all its forms, is the
one familiar to most developers, and it is in widespread use. The introduction
of agile development practices and test-driven development has also popularized
the concept of \textit{unit testing}, a form of testing in which small modules
of a program or system are tested individually.

While testing is popular and often works well, it is incomplete and informal,
and thus yields no proof that the program does what it should --- i.e.\ follows
its specification. Formal verification techniques, such as theorem proving and
\textit{model checking} (and its bounded variant), can give such proofs.
However, they suffer from complexity problems (such as \textit{incompleteness},
\textit{undecidability}) and practical issues, such as the so-called
\textit{state explosion} problem.

A relatively new approach in this area is \textit{runtime verification}, in
which the program \textit{execution} is verified against its specification, at
runtime. With the specification written in a suitably formal language, we can
verify the correctness of the program using formal approaches.

This is where this report takes off, with formal verification on one side and
testing on the other, and with runtime verification somewhere in between. In
this report, we investigate how the syntax and tooling of unit testing can be
used to improve the ease of using runtime verification.


\section{Problem Statement} \label{section-problem-statement}

In this report we are concerned with the following questions. How can runtime
verification specifications be written in a manner that uses the syntax of the
target program's programming language, and resembles the structure of unit
tests? Can we still give a formal semantics to the specification language, or a
part of it? How can we bridge the spectrum of different approaches for
verification, creating something in between the formal and informal techniques?


\section{Motivation}

Checking that a program works correctly is of great interest to software
developers. Formal verification techniques are helpful, but as mentioned above,
traditional methods can be impractical with larger programs, and verification
by testing is informal and incomplete. Runtime verification can here be a
lightweight addition to the set of verification techniques.

The specification languages used by runtime verification approaches are often
based on formal languages/formalisms (e.g.\ logic or algebra) and not written
in the target program's programming language. This means that writing the
specifications requires specific knowledge and expertise in mathematics. It
also requires mental context-switching, between writing the program and writing
the specification, and special tools to support this specialised language's
syntax. In contrast, unit testing frameworks often utilise the programming
language to great effect, and they are a common part of the software
development process. However, formal specifications allow for mathematical
reasoning and proofs about properties of the program. A combination of the
simpleness of the syntax and tool-support for testing with the mathematical
properties of formal specifications could add more certainty to the
verification in software development. If runtime verification specifications
more resembled unit tests, and were written in the target program's programming
language, it might popularise the use of runtime verification for checking the
correctness of programs.


\section{Contribution}

In this report we give a background on runtime verification and unit testing,
and describe a proof-of-concept implementation called \textit{pythonrv}.
\textit{pythonrv} is a runtime verification framework for Python. It allows the
us to write specifications, both formal and informal,
directly in Python. It makes good use of the Python decorator syntax, and gives
a clean separation between the program and the specifications, while allowing
the specifications to use the power of Python. It is the approach to formal
verification for specifications written in Python that is our main
contribution.

Naturally, there are some areas which are left out-of-scope for this report. We
give a formalization for only a subset of Python --- for specifications
constructed in a special way. We do not attempt to formalize Python. Also, we
only deal with a single thread of execution --- no multi-threading or multiple
processes --- and we do not handle real-time specifications. No performance
testing or benchmarking is done on the implementation. These limitations to
this work are of course all very interesting, and would be very suitable for
future work in this area.


\section{Report Organization}

In this chapter we have given a short introduction to and description of the
problem matter. In the next two chapters we introduce runtime verification and
unit testing, see Chapters~\ref{chapter-intro-to-rv}
and~\ref{chapter-intro-to-unit-testing}, respectively. In these chapters we
give a background to the area of program verification, and delve into the
details of specification languages, approaches to code instrumentation, and
current ideas in unit testing.

In Chapter~\ref{chapter-approach} we describe the approach we take in this work to
solve the problem stated in Section~\ref{section-problem-statement}. We
describe the syntax, instrumentation and verification techniques used in a
proof-of-concept implementation, and give a formal foundation to a subset of
the syntax. In Chapter~\ref{chapter-evaluation} we then give an evaluation of
the framework by applying it on a real-life web application.

Finally, in Chapter~\ref{chapter-conclusions} we draw some conclusions and
discuss this and potential future work.

The first time an important concept is introduced it is written in
\textit{italics}. If such a concept is not explained in the text, it has a
brief description in the Glossary, see Appendix~\ref{appendix-glossary}.
