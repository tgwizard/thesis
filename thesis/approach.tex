%================================================
%====== Chapter 4, Approach
%================================================

\pagestyle{newchap}
\chapter{Approach} \label{chapter-approach}
\lstset{language=Python}

In this report we investigate whether it is possible to do runtime verification
with specifications written in the target program's programming language,
structured similar to unit tests. To this end, there are four issues we
address:

\begin{itemize}
  \item[\textbf{Syntax}] We define the syntax and general structure of
    specification function in the next section.
  \item[\textbf{Instrumentation}] How these specifications gain access to the
    target system is described in
    Section~\ref{section-approach-instrumentation}.
  \item[\textbf{Formalization}] A subset of python, using a special structure
    and methods, is given a formal foundation, and specifications written are
    considered formal. This is described in
    Section~\ref{section-approach-formal-foundation}.
  \item[\textbf{Verification}] We have chosen to perform verification online
    and synchronous with the target system. The technique for verification
    differs for informal and formal specification functions.  See
    Sections~\ref{section-approach-verification}
    and~\ref{section-approach-formal-foundation}, respectively.
\end{itemize}

This chapter is a documentation on how we have chosen to solve these issues.
The following sections are each dedicated to one issue, and shows a
proof-of-concept of these ideas. The implementation, called \textit{pythonrv},
can be found online\footnote{\texttt{https://github.com/tgwizard/pythonrv}}.

During the development of this proof-of-concept, the biggest factor in deciding
what language to use was how it would assist in instrumentation. The language
should also be in wide use, support quick development, and have an active
testing culture.

Easy access to a non-trivial and actively used system for real-world testing
would be a plus. More on this in Chapter~\ref{chapter-evaluation}.

Python\footnote{\texttt{http://www.python.org}}, among several languages, fits
these criteria, and was chosen as the implementation language. Python is a
dynamic programming language, usually running inside a virtual machine.
Perhaps the most striking feature of Python is its use of indentation as block
delimiters. An increase in indentation starts a new block; a decrease marks the
end of the current block.

The name \textit{pythonrv} clearly reflects that it is a framework for runtime
verification in Python. The syntax of the \textit{pythonrv} specifications are
naturally heavily influenced by the fact that they are written in Python.

\textit{pythonrv} is not the first runtime verification framework for Python,
see e.g.\ \textsc{LogScope} \cite{barringer09tutorial}. But it is unique in its
use of pure Python as specification language.


\section{Syntax} \label{section-approach-syntax}

The canonical framework for doing unit testing in Python is the
\textit{unittest} framework that is included in all modern versions of Python.
Not much development has happened on it in the last years. Many new frameworks
have have spawned, such as PyUnit, Nose and py.test. They build upon the style
of unittest and mostly add new miscellaneous features, such as better test
reporting. The original structure of the unit tests is still prevalent ---
unittest builds on the xUnit style of unit testing, discussed in
Chapter~\ref{chapter-intro-to-unit-testing}.

\subsection{General Structure of a Specification}

Specifications in \textit{pythonrv} are structured as \textit{specification
functions}.

\begin{mydef}[specification function]\label{def-specification-function}
In \textit{pythonrv}, a \textit{specification function} is a Python function
describing a specification, which \textit{pythonrv} can use for verification of
the program. A specification function takes a special \textit{event} argument.
A specification \textit{monitors} points of the program, and the points being
monitored are called \textit{monitorees}. \textit{pythonrv} supports only
monitoring of function calls.
\end{mydef}

The restriction that \textit{pythonrv} only support monitoring of function
calls is not a very limiting one. For example, in Python, operators such as $+$
and $-$ are functions, and reading and writing variables can be turned into
function calls through the $property$ construct and other so-called magic
functions. See the Python documentation\footnote{Specifically
\texttt{http://docs.python.org/2/reference/datamodel.html}.} for more details.
\textit{pythonrv} thus supports monitoring of these events as well.

A specification function consists of any valid Python code. It is passed the
special \texttt{event} during verification, which gives the specification
function access to data about the current event.

In Figure~\ref{figure-specification-flow} we show what happens when a monitored
function is called. The monitoree is wrapped, so the function call first
reaches the verification code. This will do some setup and pre-processing and
then pass the event on to the specifications monitoring this function. If these
do not find any violations, the actual function is called, and the return value
is sent back to the system. As we'll see later in this section and in
Section~\ref{section-approach-instrumentation}, this is a somewhat simplified
version. Verification can, at the discretion of the specification writer, occur
after the function call.

\begin{figure}[h!]
  \begin{center}
    \begin{minipage}{0.7\textwidth}
      \begin{sequencediagram}
        \newthread{system}{System}
        \newinst{function}{Function}
        \newinst[1]{verifier}{Verifier}
        \newinst[1]{specification}{Specification}

        \begin{call}{system}{call}{verifier}{return}
          \begin{call}{verifier}{verify event}{specification}{}
          \end{call}
          \begin{call}{verifier}{call}{function}{return}
          \end{call}
        \end{call}
      \end{sequencediagram}
    \end{minipage}
  \end{center}
  \caption{The sequence for verifying a function call event.}
	\label{figure-specification-flow}
\end{figure}

The structure of a informal \textit{pythonrv} specification function is shown
in pseudo-code in Figure~\ref{figure-pseudo-spec}, and the structure of the
special \texttt{event} parameter is shown in
Figures~\ref{figure-event-functions},~\ref{figure-event-data}
and~\ref{figure-event-function-call}.

Note that the Python \texttt{assert} statement raises an
\texttt{AssertionError} exception when its argument evaluates to
\texttt{False}. This is used in \textit{pythonrv} to represent a specification
violation, see Section~\ref{section-approach-verification}.

\begin{figure}[h!]
	\begin{center}
	\begin{minipage}{0.7\textwidth}
    \begin{alltt}
\emph{import} pythonrv module
\emph{import} modules with functions to be monitored

\emph{decorate} the spec with its monitorees
\textbf{define} specification function \textit{spec}(\textbf{event}):
  \textit{# perform verification here}
  \textbf{if} bad input:
    raise an \textbf{AssertionError} exception

  \textbf{assert} properties that should be true
    \end{alltt}
	\end{minipage}
	\end{center}

  \caption{The structure of a \textit{pythonrv} specification.}
	\label{figure-pseudo-spec}
\end{figure}

\begin{figure}[h!]
	\begin{center}
	\begin{minipage}{0.7\textwidth}
	\begin{lstlisting}
# rv.py
# the Event class has some special
# functions
class Event:
  def finish(success):
    if success:
      self.success()
    else:
      self.failure()
  def success():
    # remove this spec from future
    # verification
  def failure():
    # fail the spec by raising an
    # exception
    raise AssertionError("Violation")

  def next(new_spec):
    # add new_spec to be used in
    # verification - on the next
    # event only
  def next_called_shoud_be(monitoree):
    # on the next Event, the function
    # called should be monitoree. Verify
    # this
	\end{lstlisting}
	\end{minipage}
	\end{center}

  \caption{The structure of the special \textit{event} argument passed to
    \textit{pythonrv} specification functions. The example is written in
    pseudo-python.}
	\label{figure-event-functions}
\end{figure}

\begin{figure}[h!]
	\begin{center}
	\begin{minipage}{0.7\textwidth}
	\begin{lstlisting}
# rv.py, cont'd
# an Event object is populated with data
event = Event()
# list of previous events
event.history
# the event that occurred before this
event.prev = history[-1]

# map (name => FunctionCall) for functions
# monitored by the spec
event.fn
# the monitored function that generated
# this event
event.called_function
	\end{lstlisting}
	\end{minipage}
	\end{center}

  \caption{The structure of the data in the \textit{event} argument passed to
    \textit{pythonrv} specification functions. The example is written in
    pseudo-python.}
	\label{figure-event-data}
\end{figure}

\begin{figure}[h!]
	\begin{center}
	\begin{minipage}{0.7\textwidth}
	\begin{lstlisting}
# rv.py, cont'd
# the FunctionCall class
class FunctionCall:
  def next(new_spec):
    # same as for Event but only for Events
    # where this function was called

fc = FunctionCall()
# the name of the function
fc.name
# True if the function was called in this event
fc.called
# the input arguments for the function
fc.inputs
# the output arguments for the function
fc.outputs
# the function's return value
fc.result
	\end{lstlisting}
	\end{minipage}
	\end{center}

  \caption{The structure of the function call structures in the \textit{event}
    argument passed to \textit{pythonrv} specification functions. The example
    is written in pseudo-python.}
	\label{figure-event-function-call}
\end{figure}



\clearpage
\subsection{Specification Functions by Example}
\lstset{language=Python,numbers=left}

\begin{figure}[h!]
	\begin{center}
	\begin{minipage}{0.5\textwidth}
	\lstinputlisting{figures/syntax_example_1.py}
	\end{minipage}
	\end{center}

  \caption{A very simple \textit{pythonrv} specification function.}
	\label{figure-syntax-example-1}
\end{figure}

We now proceed with a few simple examples to show how informal
\textit{pythonrv} specifications can be written. More realistic examples are
shown in Chapter~\ref{chapter-evaluation}.

The first example, in Figure~\ref{figure-syntax-example-1}, shows the basics of
a \textit{pythonrv} specification function. The specification just verifies,
through the \texttt{assert} statement, that the first input to the monitored
function is always greater than zero.

On line 1 we import the \texttt{rv} module from the \texttt{pythonrv} package,
which gives us access to the runtime verification features of
\textit{pythonrv}. We specify that the specification should monitor the
function \texttt{fib} in the module \texttt{fibmodule} on line 4, and that
whenever \texttt{fib} is called, the specification should be verified. The
monitoree \texttt{fib} is, locally to the specification function, aliased as
\texttt{func}. The instrumentation is also done on line 4 by using the
\textit{function decorator} \texttt{rv.monitor} (see
  Section~\ref{section-approach-instrumentation} for an explanation of function
decorators and \texttt{rv.monitor}.)

On line 5 we define the specification as an ordinary Python function called
\texttt{spec}, taking one argument, \texttt{event}. On line 6, the array of
input arguments used to call \texttt{fib} is accessed to check that the first
argument is greater than zero.

The specification function in Figure~\ref{figure-syntax-example-1} will be
called upon every invocation to \texttt{fibmodule.fib}.

\begin{figure}[h!]
	\begin{center}
	\begin{minipage}{0.7\textwidth}
	\lstinputlisting{figures/syntax_example_2.py}
	\end{minipage}
	\end{center}

  \caption{A \textit{pythonrv} specification that monitors two functions,
    \texttt{mymodule.foo} and \texttt{mymodule.bar}.}
	\label{figure-syntax-example-2}
\end{figure}

In the second example, Figure~\ref{figure-syntax-example-2}, we show how a
specification function can monitor two functions. The specification function
will be called whenever either of the monitored functions are called. The
example specification verifies that the calls to the two functions alternate;
that no two calls to \texttt{foo} occur without a call to \texttt{bar} in
between, and vice versa. The first call has to be to \texttt{foo}.

Which function was called can be determined from the \texttt{event} argument,
as is done on lines 7 and 14. It is the \texttt{called} attribute of a function
in the \texttt{event.fn} structure that allows for this.

We also show in the example how the specification can access a history of
previous events - events that it has handled in the past.
\texttt{event.history} is a list of all events that has occurred that this
specification monitors. The last element is the current event, and the
next-to-last element is the previous element, which can also be accessed as
\texttt{event.prev}.

\begin{figure}[h!]
	\begin{center}
	\begin{minipage}{0.7\textwidth}
	\lstinputlisting{figures/syntax_example_3.py}
	\end{minipage}
	\end{center}

  \caption{A more complex example, utilizing the \texttt{event.next} function.}
	\label{figure-syntax-example-3}
\end{figure}

In the last example, Figure~\ref{figure-syntax-example-3}, we show a more
advanced specification, in which the \texttt{next} function of the
\texttt{event} argument is used. \texttt{event.next} allows the specification
function to add more specification functions (possibly implemented as closures
or lambdas) to be executed when the next event occurs.

The specification function monitors three functions, \texttt{foo}, \texttt{bar}
and \texttt{baz}, and makes sure that \texttt{foo} is called first, then any
number calls to \texttt{bar} with the first argument as \texttt{True}, and then
finally a call to \texttt{baz}. After that, any calls are allowed --- the
specification function will not be used in verification any longer. So, the
call event sequence $(\texttt{foo()}, \texttt{baz(True)}, \texttt{baz(True)},
\texttt{baz(True)}, \texttt{bar()})$ is accepted, but $\texttt{baz(True)},
\texttt{bar()}$ is not.

The specification is split into two parts, \texttt{spec} and \texttt{followup}.
\texttt{spec} verifies that \texttt{foo} is called first, and then that
\texttt{followup} should be verified on the following event. After that, on
line 11, \texttt{spec} notifies that it has finished its verification, and
should not be used in the future --- it ``unsubscribes'' from future events.

On line 10 the function \texttt{followup} is added to be executed on the next
event. We use \texttt{followup} to accept any calls to \texttt{baz} (with the
first argument set to \texttt{True}). \texttt{followup} finishes its
verification when \texttt{bar} has been called.

Since \texttt{followup} is added to the verification machinery through the
\texttt{event.next} function --- as a ``oneshot'' specification function --- it
needs to add itself using \texttt{next} for verification on subsequent events.
We do this on line 24.



\subsection{Customization}

The examples above show the main capabilities of \textit{pythonrv}
specifications. However, some details can be customized through the
\texttt{@rv.spec} decorator, which accepts the following parameters (default
values in parentheses):

\begin{itemize}
  \item[\texttt{when}] (\texttt{rv.PRE}). This argument allows the
    specification writer to determine when the specification should be
    verified, before or after the monitorees are called. The valid arguments
    \texttt{rv.PRE} and \texttt{rv.POST}, respectively. Only when
    \texttt{when=rv.POST} does the specification function have access to the
    monitorees' return values.

  \item[\texttt{history\_size}] (\texttt{2}). Specifies how much event history
    should be remembered for the specification.
    \texttt{rv.INFINITE\_HISTORY\_SIZE} can be used to remember all previous
    events, although this could naturally be a big drain on the available
    memory resources.

  \item[\texttt{level}] (\texttt{rv.ERROR}). Specifies how severe or important
    the specification is. Any value can be used; \texttt{DEBUG}, \texttt{INFO},
    \texttt{WARNING}, \texttt{ERROR}, \texttt{CRITICAL} are supplied by
    \texttt{pythonrv}. With these levels we can have more control over what
    should happen when a specification violation occurs, see
    Section~\ref{section-approach-verification}.
\end{itemize}


\section{Instrumentation} \label{section-approach-instrumentation}

The previous section showed how \textit{pythonrv} specification functions can
be written. This section will describe how these functions can jack themselves
into the ordinary control flow of the program and gain access to the function
call events and their arguments and associated state.

Instrumentation is done through the \texttt{rv.monitor} \textit{function
decorator} in \textit{pythonrv}. A Python function decorator is similar to
attributes in~.Net and annotations in Java. It is essentially a function that
takes a function as a parameter, possibly modifies it, or uses it in some way
(decorates it), and then returns it (or another function). This is used
throughout Python to, for instance, turn functions into static or class
methods. In Figure~\ref{figure-function-decorator} we show an example function
decorator definition, and in Figure~\ref{figure-function-decorator-usages} we
show the syntax for using it.

\begin{figure}[h!]
	\begin{center}
	\begin{minipage}{0.7\textwidth}
	\lstinputlisting{figures/function_decorator.py}
	\end{minipage}
	\end{center}

	\caption{An example of how to define a function decorator.}
	\label{figure-function-decorator}
\end{figure}

\begin{figure}[h!]
	\begin{center}
	\begin{minipage}{0.7\textwidth}
	\lstinputlisting{figures/function_decorator_usages.py}
	\end{minipage}
	\end{center}

	\caption{An example of how to use the function decorator from
	Figure~\ref{figure-function-decorator}.}
	\label{figure-function-decorator-usages}
\end{figure}

The decorator in Figure~\ref{figure-function-decorator} caches the return
values for functions on which it is applied, so that future calls to a
decorated function with the same arguments will short-circuit and return the
cached value directly, never entering the function itself. As is shown in
Figure~\ref{figure-function-decorator-usages}, the decorator is something that
can be attached with a special syntax, using the \texttt{@} operator. The
decorated function does not need any further modifications.

\texttt{rv.monitor} is a decorator similar to the \texttt{cache} decorator
above. It takes to set of arguments, first what functions should be monitored
(the monitorees), and second the specification function. However, the
\texttt{rv.monitor} decorator does not modify the function which it decorates,
which here is the specification function, but instead finds the monitorees and
decorates them with a dynamically generated function \texttt{wrapper}. The code
in Figure~\ref{figure-instrumentation-overview} illustrates how
\textit{pythonrv} does instrumentation.

\begin{figure}[h!]
	\begin{center}
	\begin{minipage}{0.7\textwidth}
	\begin{lstlisting}
# rv.py
def monitor(monitorees, specification):
	for monitoree in monitorees:
		# define a wrapper for each monitoree
		def wrapper(*args, **kwargs)
			event = create_event(...)

			# call specification
			specification(event)

			# call the actual function - the
			# monitoree
			return monitoree(*args, **kwargs)

		# overwrite the monitoree in its container
		container = get_container(monitoree)
		setattr(container, monitoree.name, wrapper)

monitor(myfunc, myspec)
myfunc(x, y, z) # will call myspec first
	\end{lstlisting}
	\end{minipage}
	\end{center}

	\caption{An overview of the \textit{pythonrv} instrumentation process,
		written in pseudo-Python. This is just for illustrative purposes and not
		how \textit{pythonrv} actually does the instrumentation.}
	\label{figure-instrumentation-overview}
\end{figure}

The instrumentation in \textit{pythonrv} works as follows. First, a wrapper
function is defined for each function to be monitored (for each monitoree).
This wrapper function's main purpose is to call the specifications attached to
the monitored function, and then call the monitored function itself. The
wrapper also does some argument copying and such, to prevent side-effects in
the specifications from interfering with the monitored function.

Usually in Python, functions belong to a parent \textit{container}, such as a
class, an object, or a module\footnotemark. In
Figure~\ref{figure-function-decorator-usages} the functions \texttt{fact} and
\texttt{fact2} belong to the module \texttt{test} (the module's name is the
same as that of the file containing the code). These containers are essentially
dictionaries (\textit{dicts} in Python parlance) of key-value pairs, where the
keys in this case are function names and the values are objects representing
the function code. (There are other types of values in these containers as
well, which we can ignore).

\footnotetext[4]{This is not true of closures --- functions defined inside other
  functions. These functions cannot be directly referenced or modified from
  outside the defining function. \textit{pythonrv} does not (as of writing)
  support monitoring of closures.}

So in the \textit{pythonrv} instrumentation process, the parent container of
each monitoree is extracted, and the reference to the monitoree is overwritten
with a reference to the wrapper. The wrapper still has a reference to the
monitoree.

The implementation of the instrumentation code in \textit{pythonrv} is more
optimized than what is shown in Figure~\ref{figure-instrumentation-overview}.
For instance, each monitoree only ever have one wrapper function, independent
of how many specifications want to monitor it. The wrappers then iterate over
the relevant specifications, passes the control (and an event) to them, and
then runs the monitoree. As we stated in section~\ref{section-approach-syntax},
a specification can either be verified before or after the function call. The
wrapper function handles this.



\section{Verification of Informal Specifications}
\label{section-approach-verification}

In \textit{pythonrv}, verification of informal specifications means just
executing them. The specification functions are valid python functions, and
executing them on the appropriate events, providing access to the corresponding
data, verifies the specification they represent.

Specification functions notify verification violations, that the specifications
are not followed, by raising exceptions of the type \texttt{AssertionError}.
These exceptions are raised when the \texttt{assert} statement fails. They can
also be raised manually: \texttt{raise AssertionError('error message')}.

The verification is performed online, during the program execution.
Specifications are verified for all calls to function they monitor unless they
explicitly remove themselves by calling one of \texttt{event.finish},
\texttt{event.success} and \texttt{event.failure} (described in
Section~\ref{section-approach-syntax}).

Whenever a specification violation occurs, and an \texttt{AssertionError} is
raised, it is passed to an \textit{error handler}. There are two built-in error
handlers. One, the default, re-raises the \texttt{AssertionError} exception,
and thus causes it to propagate up through the call stack. If the exception
is not caught and suppressed by code higher up the call stack, the program will
crash. \texttt{AssertionError} are supposed to be raised when something has
gone wrong and the operation should halt, so it is bad coding practice to
suppress them.

A second error handler just logs the error message, using the standard Python
logging module.

We can customize what should happen when verification fails for a
specification. We do this by creating an error handler and passing it to the
\texttt{rv.configure} function. Any object with a \texttt{handle} method that
takes two arguments, the specification \texttt{level} and a list of errors, is
a valid error handler.

In Figure~\ref{figure-configure-error-handler} we use a simple error handler
that only propagates the verification errors, and thus crashes the program, for
specifications with level \texttt{rv.CRITICAL}. A specification with
\texttt{level} set to \texttt{rv.CRITICAL} is also shown.

\begin{figure}[h!]
	\begin{center}
	\begin{minipage}{0.7\textwidth}
	\lstinputlisting{figures/configure_error_handler.py}
	\end{minipage}
	\end{center}

  \caption{How to use a custom error handler.}
	\label{figure-configure-error-handler}
\end{figure}

The current verification approach in \textit{pythonrv} is to perform it online.
This obviously affects the performance of the program under test. Offline
verification could be used to mitigate this, removing all overhead but for the
required recording layer. To do offline verification in \textit{pythonrv} the
events and their associated data would need to be saved (serialized) and
replayed outside the context of the running program. This is suggested as an
area for future work in Chapter~\ref{chapter-conclusions}.





%================================================
%====== Formal Foundation
%================================================

\section{Formal Foundation} \label{section-approach-formal-foundation}

The purpose of a formal foundation for a verification approach is to reason
mathematically about the system. The specifications used for verification need
some sort of mathematical representation. A model of the system, a system
model, is also required. In this case the system model is the sequence of that
occur during the program's execution, with their associated data (arguments,
object state, global state, etc.). This is described in more detail in
Section~\ref{section-approach-formal-foundation-semantics}

\textit{pythonrv} specifications are written as ordinary Python functions so we
have to formalize them.  The Python programming language is rather informal -
one implementation of it, CPython, serves as the reference implementation.
There are no other specifications or formal semantics for Python\footnote{The
development of Python is organized mainly through the Python Enhancement
Proposal (PEP) process. PEPs are design documents for new features,
informally describing their rationales and how they work.}.

One way to go around this is to define formal semantics for a subset of Python,
which is done in the following sections. This leads to a way to reason
mathematically with and about specifications written in this subset, which we
describe in the next section.




%================================================
%====== Formal Foundation - Python Subset
%================================================

\subsection{\textit{pythonrv} Language for Formal Specification Functions}
\label{section-approach-python-subset}
\lstset{language=Python,numbers=left}

We define the formal specification language in \textit{pythonrv} as a subset of
Python. This allows us to write \textit{formal specification functions}.

\begin{mydef}[primitive formal specification functions]
\label{def-primitive-formal-specification-function}
A \textit{primitive formal specification function} is one of four basic
function structures: The \textit{assert}, \textit{next}, \textit{if-then} and
\textit{if-then-else} functions, shown in
Figure~\ref{figure-basic-formal-specification-functions}. The expression
\textit{E} used in the \textit{assert}, \textit{if-then} and
\textit{if-then-else} functions (lines 2, 14 and 21 in
Figure~\ref{figure-basic-formal-specification-functions}) denotes any
idempotent, immutable, valid Python boolean expression (a boolean expression is
one which evaluates to either true or false). Evaluating an expression should
not change the program state --- it should be idempotent and immutable --- or
the act of verifying the program against its specifications could affect the
success of the verification. If the expression needs to access any properties
of the \texttt{event} argument, the expression needs to be of the form
\texttt{lambda event: E}.
\end{mydef}

\begin{figure}[h!]
	\begin{center}
	\begin{minipage}{0.7\textwidth}
	\begin{lstlisting}
def spec():
  a = make_assert(E)
  tail = # optional composition point 'tail'
  return a + tail

def next():
  x = # required composition point 'next'
  n = make_next(x)
  tail = # optional composition point 'tail'
  return n + tail

def if_then():
  then = # required composition point 'then'
  i = make_if(E, then)
  tail = # optional composition point 'tail'
  return i + tail

def if_then_else():
  then = # required composition point 'then'
  els = # required composition point 'else'
  i = make_if(E, then, els)
  tail = # optional composition point 'tail'
  return i + tail
	\end{lstlisting}
	\end{minipage}
	\end{center}

	\caption{The four basic formal specification functions.}
	\label{figure-basic-formal-specification-functions}
\end{figure}

Primitive formal specification functions can be composed together to form more
complex specifications. Composition is done at \textit{composition points},
using the \textit{composition operator} $\circ$. We describe composition in
Section~\ref{section-approach-composition}.

\begin{mydef}[composition points]\label{def-composition-point}
Formal specification functions have \textit{composition points} where they can
be combined with other formal specification functions. A formal specification
function can have zero or more composition points. A composition point can be
\textit{open} or \textit{closed} --- open composition points can be used in
composition, while closed have already been used. Composition points can be
\textit{required} or \textit{optional}.
\end{mydef}

\begin{mydef}[complete and incomplete specifications]
\label{def-composition-complete}
A formal specification function can be either \textit{complete} or
\textit{incomplete}. A complete formal specification function is a
specification that can be used by \textit{pythonrv} for verification. An
incomplete formal specification function cannot be used for verification yet,
but can, with composition, become complete. Complete formal specification
functions have no open required composition points; incomplete formal
specification functions have at least one. Only the \textit{assert} primitive
specification function is complete --- the other can become complete through
composition.
\end{mydef}

\begin{mydef}[the composition operator]\label{def-composition-operator}
Composition is described by the \textit{composition operator} $\circ$. Let $f$,
$g$ and $h$ be formal specification functions. Let $f$ have one composition
point, $g$ two, labeled $a$ and $b$, respectively, and $h$ none. An example
composition could be: $s = f \, \circ \, ((g \, \circ_{a} \, h) \, \circ_{b} \,
h)$. $s$ would be a complete formal specification function, as it is composed
of formal specification functions, and it has no open required composition
points (or optional ones). We define what composition actually means below.
\end{mydef}

\begin{mydef}[automata]\label{def-automata}
Every formal specification function $s$ can be represented by a deterministic
finite automata $A(s) = (Q, P, q_0, S, F)$. Each such automata consists of a
set of states $Q$ and transitions $(a, b, E) \in P$ between them, where $a$ is
the start state, $b$ is the end state, $a,b \in Q$ and $E$ is a
\textit{transition label}. The transition label is described in detail in
Section~\ref{section-approach-formal-foundation-semantics} An automata has one
\textit{initial state} $q_0 \in Q$, a set of \textit{success states} $S
\subseteq Q$ and a set of \textit{fail states} $F \subseteq Q$. $S$ and $F$ are
disjoint. Success states are depicted as \textit{accepting states} in the
automata. Fail states are usually called \textit{fail}, $f$, etc.
\end{mydef}

The four basic formal specification functions correspond to simple,
deterministic, finite automata, sketches of which are depicted in
Figure~\ref{figure-basic-formal-specification-automata}. Please note that the
automata in Figure~\ref{figure-basic-formal-specification-automata} are only
sketches (as can be seen by the squiggly cloud-shapes), and are only supposed
to give an overview of how they work. They are defined properly in the next
section.

\begin{figure}[h!]
	\begin{center}
	\begin{minipage}{0.7\textwidth}
	\begin{lstlisting}
from pythonrv import rv
from pythonrv.formalrv import (formal_spec,
  make_assert, make_next, make_if)

@rv.monitor(monitorees)
@formal_spec
def spec():
  # spec is a formal specification function
  # - assert, next, if-then or if-then-else
	\end{lstlisting}
	\end{minipage}
	\end{center}

  \caption{The boilerplate code for formal \textit{pythonrv} specification
  functions that allow for verification and instrumentation to take place.}
	\label{figure-formal-boilerplate}
\end{figure}

A complete and verifiable formal specification function also needs some
boilerplate, shown in Figure~\ref{figure-formal-boilerplate}. Lines 1 and 5 are
the same as for informal specification functions --- we import the
\textit{pythonrv} \texttt{rv} package, and annotate our specification with what
function we want to monitor. On lines 2 and 3 we import functions required for
formal \textit{pythonrv} specifications, and line 6 is the crucial difference,
where we say that this specification function should use formal verification.


%%%%%%%%%%%%%%%%%%%%%%%%
%% Automata for the basic formal specification functions
%%%%%%%%%%%%%%%%%%%%%%%%
\begin{figure}[h!]
	\begin{minipage}{0.45\textwidth}
		\centering
		\begin{tikzpicture}[->,shorten >=1pt,node distance=2cm,on
      grid,auto,scale=.5,semithick,>=stealth]
			\node[state,initial] (q0) {$q_0$};
			\node[state,accepting] (q1) [right of=q0] {$q_1$};
      \node[state] (f) [below right of=q0] {\textit{f}};
			\path
				(q0) edge node {$E$} (q1)
        (q0) edge node {$\neg E$} (f);
		\end{tikzpicture}
    \subcaption{A generalized \textit{assert} specification, asserting the
      expression $E$.}
	\end{minipage}
  ~
	\begin{minipage}{0.45\textwidth}
		\centering
		\begin{tikzpicture}[->,shorten >=1pt,node distance=2.5cm,on
      grid,auto,scale=.5,semithick,>=stealth]
			\node[state,initial] (q0)	{$q_0$};
			\node[cloud, cloud puffs=15.7, cloud ignores aspect, align=center, draw] (cloud) [right of=q0] {$A(next)$};
			\path
				(q0) edge node {$\top$} (cloud);
		\end{tikzpicture}
    \subcaption{A template for a \textit{next} specification.}
	\end{minipage}

  \bigskip

	\begin{minipage}{0.45\textwidth}
		\centering
		\begin{tikzpicture}[->,shorten >=1pt,node distance=2.5cm,on
      grid,auto,scale=.5,semithick,>=stealth]
			\node[state,initial] (q0)	{$q_0$};
			\node[cloud, cloud puffs=15.7, cloud ignores aspect, align=center, draw] (then) [right of=q0] {$A(then)$};
			\node[state,accepting] (q1) [below of=q0]	{$q_1$};
			\path
				(q0) edge node {$E$} (then)
        (q0) edge node {$\neg E$} (q1);
		\end{tikzpicture}
    \subcaption{A template for an \textit{if-then} specification.}
	\end{minipage}
  ~
	\begin{minipage}{0.45\textwidth}
		\centering
		\begin{tikzpicture}[->,shorten >=1pt,node distance=2.5cm,on
      grid,auto,scale=.5,semithick,>=stealth]
			\node[state,initial] (q0)	{$q_0$};
			\node[cloud, cloud puffs=15.7, cloud ignores aspect, align=center, draw] (then) [right of=q0] {$A(then)$};
			\node[cloud, cloud puffs=15.7, cloud ignores aspect, align=center, draw] (else) [below of=then] {$A(else)$};
			\path
				(q0) edge node {$E$} (then)
        (q0) edge node {$\neg E$} (else);
		\end{tikzpicture}
    \subcaption{A template for an \textit{if-then-else} specification.}
	\end{minipage}

  \caption{Sketches of automata for the four basic formal specification
    functions. The squiggles around $A(next)$, $A(then)$ and $A(else)$ denote
    that some composition is required there.}
	\label{figure-basic-formal-specification-automata}
\end{figure}
%%%%%%%%%%%%%%%%%%%%%%%%
%%%%%%%%%%%%%%%%%%%%%%%%
%%%%%%%%%%%%%%%%%%%%%%%%


These preliminaries allow us to define primitive formal specification functions
mathematically as automata, and formal rules on how to compose them.



\subsection{Automata Representation and Rules for Composition}
\label{section-approach-composition}
\lstset{language=Python,numbers=none}

With each formal specification function represented as an automaton, we can use
composition to build more complex and interesting specifications. Composition
is defined inductively, preserving the semantics, which we define in
Section~\ref{section-approach-formal-foundation-semantics}.

Notation: Let $a_E$ be an \textit{assert} specification asserting the
expression $E$, let $n$ be a \textit{next} specifications and let $i_E$ be an
\textit{if-then} or \textit{if-then-else} specification with the expression $E$
guarding the \textit{then} composition point. Let $s$ be any specification.
Appending subscripts and $'$ denotes different specifications or expressions of
the same type. The $E$ subscript can be omitted if irrelevant to the task at
hand.

\subsubsection{The \textit{assert} specification, and composition with the
\textit{tail} composition point}

The \textit{assert} specifications are the only primitive formal specification
functions that are complete, without the need to compose them with other
specifications. The automata $A(a_E)$ for a standalone \textit{assert}
specification $a_E$, depicted in
Figure~\ref{figure-basic-formal-specification-automata}~(a), is defined as:

\medskip
\[
  \begin{array}{rcl}
    A(a_E) & = & (\{q_0, q_1, f\}, \{(q_0, q_1, E), (q_0, f, \neg E)\}, q_0, \{q_1\}, \{f\})
  \end{array}
\]
\medskip

If $\neg E$ is true, the fail state $f$ will be reached, and the specification
has been violated.

To make future composition definitions easier we introduce the concept of
\textit{initial transitions}:

\begin{mydef}[initial transitions]\label{def-initial-transitions}
The set of \textit{initial transitions} $T$, $(q_0, q', E) \in T$,
$T \subseteq P$ consists of those transitions in $P$ that start from the
initial state $q_0$ of the automata, go to some state $q'$, and have some label
$E$. Below we will sometimes split up the transitions $P$ of automata into two
parts, the initial transitions $T$, and all other transitions $R$: $P = T \cup
R$ and $T \cap R = \emptyset$.
\end{mydef}

\textit{assert} specifications also have, together with all primitive formal
specification functions and compositions thereof, at least one open composition
point, the \textit{tail} composition point. Compositions using the
\textit{tail} composition point are commutative: $s \, \circ_{tail} \, s' = s'
\, \circ_{tail} \, s$. Composition using the \textit{tail} composition point
is essentially just a merge of the initial states of the two specifications,
making the initial transitions of both specifications go out from the same
state $q_0$, and merging the success states and fail states of the automata.
Given two specifications $s$ and $s'$, where:

\medskip
\[
  \begin{array}{rcl}
    A(s) & = & (Q_s, T_s \cup R_s, q_{0s}, S_s, F_s) \\
   A(s') & = & (Q_{s'}, T_{s'} \cup R_{s'}, q_{0s'}, S_{s'}, F_{s'})
  \end{array}
\]
\medskip

Then:

\medskip
\[
  \begin{array}{rcl}
    A(s \circ_{tail} s') & = & (Q, P, q_0, S_s \cup S_{s'}, F_s \cup F_{s'}) \\
                       Q & = & \{q_0\} \cup Q_s \cup Q_{s'} - \{q_{0s}, q_{0s'}\} \\
                       P & = & T \cup R_s \cup R_{s'} \\
                       T & = & \{(q_0, q', E) \, | \, (q, q', E) \in (T_s \cup T_{s'})\} \\
  \end{array}
\]
\medskip

An example for composing two \textit{assert} specifications is shown in
Figure~\ref{figure-assert-composition-example}. The resulting automata is
unnecessarily complex, and can be simplified to a smaller automata with the
same semantics, as seen in Figure~\ref{figure-assert-composition-simplified}.


%%%%%%%%%%%%%%%%%%%%%%%%
%% Composition, assert o s, symbol-python-automata
%%%%%%%%%%%%%%%%%%%%%%%%
\begin{figure}[h!]
	\begin{minipage}{0.90\textwidth}
		\centering
    \[
      \begin{array}{rcl}
        A(a_E) & = & (\{q_x, q_y, f\}, \{(q_x, q_y, E), (q_x, f, \neg E)\}, q_x, \{q_y\}, \{f\}) \\
    A(a'_{E'}) & = & (\{q_z, q_w, f'\}, \{(q_z, q_w, E') ,(q_z, f', \neg E')\}, q_z, \{q_w\}, \{f'\})
      \end{array}
    \]
		\centering
    \[
      \begin{array}{rcl}
        A(a_E \circ_{tail} a'_{E'}) & = & (Q, P, q_0, S, F) \\
                                 Q & = & \{q_0, q_y, f, q_w, f'\} \\
                                 P & = & \{(q_0, q_y, E), (q_0, f, \neg E), (q_0, q_w, E'), (q_0, f', \neg E')\} \\
                                 S & = & \{q_y, q_w\} \\
                                 F & = & \{f, f'\}
      \end{array}
    \]
	\end{minipage}

  \bigskip

	\begin{minipage}{0.45\textwidth}
		\centering
    \begin{lstlisting}
def s():
  return make_assert(E) +
         make_assert(E')
    \end{lstlisting}
	\end{minipage}
  ~
	\begin{minipage}{0.45\textwidth}
		\centering
		\begin{tikzpicture}[->,shorten >=1pt,node distance=2cm,on
      grid,auto,scale=.5,semithick,>=stealth]
			\node[state,initial] (q0) {$q_0$};
			\node[state,accepting] (qy) [below right of=q0] {$q_y$};
			\node[state] (f) [right of=qy] {$f$};
			\node[state,accepting] (qw) [below left of=q0] {$q_w$};
			\node[state] (f') [left of=qw] {$f'$};
			\path
        (q0) edge node [below left] {$E$} (qy)
        (q0) edge node [below right] {$E'$} (qw)
        (q0) edge node [above right] {$\neg E$} (f)
        (q0) edge node [above left] {$\neg E'$} (f');
		\end{tikzpicture}
	\end{minipage}
  \caption{An example showing a composition of an \textit{assert} specification
    $a_E$ with another \textit{assert} specification $a'_{E'}$. Only the
    resulting $A(a_E \, \circ_{tail} \, a'_{E'})$ automata is shown. $q_y$ and
    $q_w$ are success states; $f$ and $f'$ are fail states.}
	\label{figure-assert-composition-example}
\end{figure}

\begin{figure}[h!]
	\begin{minipage}{0.45\textwidth}
		\centering
    \[
      (\{q_0, q_1, f\}, \{(q_0, q_1, E \wedge E'), (q_0, f, \neg E \vee \neg E')\}, q_0, \{q_1\}, \{f\})
    \]
	\end{minipage}
	\begin{minipage}{0.45\textwidth}
		\centering
		\begin{tikzpicture}[->,shorten >=1pt,node distance=3cm,on
      grid,auto,scale=.5,semithick,>=stealth]
			\node[state,initial] (q0) {$q_0$};
			\node[state,accepting] (q1) [right of=q0] {$q_1$};
			\node[state] (f) [below right of=q0] {f};
			\path
				(q0) edge node {$E \wedge E'$} (q1)
        (q0) edge node {$\neg E \vee \neg E'$} (f);
		\end{tikzpicture}
	\end{minipage}
  \caption{A simplified version of the automata from
    Figure~\ref{figure-assert-composition-example}, semantically identical. The
  success states have been merged into $q_1$, and the fail states into $f$.}
	\label{figure-assert-composition-simplified}
\end{figure}

%%%%%%%%%%%%%%%%%%%%%%%%
%%%%%%%%%%%%%%%%%%%%%%%%
%%%%%%%%%%%%%%%%%%%%%%%%


\subsubsection{The \textit{next} specification}

The \textit{next} specification functions are the specifications that deal with
time.  \textit{next} specifications have two composition points: one
appropriately called \textit{next}, which is required, and one called
\textit{tail}, which is optional. Composition using the \textit{tail}
composition point was described above.

Composition with the \textit{next} composition point, $n \, \circ_{next} \, s$
with $A(s) = (Q, P, q_0, S, F)$ is as follows:

\medskip
\[
  \begin{array}{rcl}
    A(n \circ_{next} s) & = & (\{q_0'\} \cup Q, T \cup P, q_0', S, F) \\
               T  & = & \{(q_0', q_0, \top)\}
  \end{array}
\]
\medskip

This is illustrated in Figure~\ref{figure-next-composition-s}.

Also note that composition using the \textit{next} composition point is
associative, as shown in Figure~\ref{figure-next-associative}.



%%%%%%%%%%%%%%%%%%%%%%%%
%% Composition, next o s, symbol-python-automata
%% Associativity of the next composition point
%%%%%%%%%%%%%%%%%%%%%%%%
\begin{figure}[h!]
	\begin{minipage}{0.20\textwidth}
		\centering
    \[
      s' = n \, \circ_{next} \, s
    \]
	\end{minipage}
  ~
	\begin{minipage}{0.35\textwidth}
		\centering
    \begin{lstlisting}
def s'():
  return make_next(s)
    \end{lstlisting}
	\end{minipage}
  ~
	\begin{minipage}{0.4\textwidth}
		\centering
		\begin{tikzpicture}[->,shorten >=1pt,node distance=2.5cm,on
      grid,auto,scale=.5,semithick,>=stealth]
			\node[state,initial] (q0')	{$q_0'$};
			\node[cloud, cloud puffs=15.7, cloud ignores aspect, align=center, draw] (cloud) [right of=q0'] {$A(s)$};
			\path
				(q0') edge node {$\top$} (cloud);
		\end{tikzpicture}
	\end{minipage}
  \caption{The composition of a \textit{next} specification with any
    specification $s$, using the \textit{next} composition point.}
	\label{figure-next-composition-s}
\end{figure}

\begin{figure}[h!]
	\begin{minipage}{0.9\textwidth}
		\centering
    \[
      \begin{array}{rcl}
        (n \, \circ_{next} \, s_1) \, \circ_{tail} \, (n' \, \circ_{next} \, s_2)
        & = & n \, \circ_{next} \, t \\
        t & = & s_1 \, \circ_{tail} \, s_2
      \end{array}
    \]
	\end{minipage}

  \bigskip

	\begin{minipage}{0.45\textwidth}
		\centering
    \begin{lstlisting}
def s():
  return make_next(s1) +
    make_next(s2)

# equivalent to
def s():
  return make_next(t)

def t(event):
  return s1() + s2()
    \end{lstlisting}
	\end{minipage}
  ~
	\begin{minipage}{0.45\textwidth}
		\centering
		\begin{tikzpicture}[->,shorten >=1pt,node distance=2.5cm,on
      grid,auto,scale=.5,semithick,>=stealth]
			\node[state,initial] (q0)	{$q_0$};
			\node[cloud, cloud puffs=15.7, cloud ignores aspect, align=center, draw] (s1) [right of=q0] {$A(s_1)$};
			\node[cloud, cloud puffs=15.7, cloud ignores aspect, align=center, draw] (s2) [below right of=q0] {$A(s_2)$};
			\path
				(q0) edge node {$\top$} (s1)
				(q0) edge node {$\top$} (s2);
		\end{tikzpicture}

    \bigskip

		\centering
		\begin{tikzpicture}[->,shorten >=1pt,node distance=2.5cm,on
      grid,auto,scale=.5,semithick,>=stealth]
			\node[state,initial] (q0) {$q_0$};
			\node[cloud, cloud puffs=15.7, cloud ignores aspect, align=center, draw] (t) [right of=q0] {$A(t)$};
			\path
				(q0) edge node {$\top$} (t);
		\end{tikzpicture}
	\end{minipage}
  \caption{The associativity of the \textit{next} composition point in a
  \textit{next} formal specification function.}
	\label{figure-next-associative}
\end{figure}
%%%%%%%%%%%%%%%%%%%%%%%%
%%%%%%%%%%%%%%%%%%%%%%%%
%%%%%%%%%%%%%%%%%%%%%%%%


\subsubsection{The \textit{if-then} and \textit{if-then-else} specifications}

\textit{if-then} specifications have two composition points: one required,
\textit{then}, and one optional, \textit{tail}. \textit{if-then-else}
specifications have an additional required composition point \textit{else}.
Composition using the \textit{tail} composition point was described above.

In an \textit{if-then-else} specification $i_E$ we consider the expression $E$
as a guard for the \textit{then} composition point, and $\neg E$ as a guard
for the \textit{else} composition point. In \textit{if-then} specifications,
the $\neg E$ guard goes to a success state.

The composition essentially becomes to add the guard $E$ to the labels of all
initial transitions for the automata at the \textit{then} composition point,
and $\neg E$ to the labels of all initial transitions for the automata at the
\textit{else} composition point. The guards, $E$ and $\neg E$ are added as
parts of a conjunctive.

Let $s_1$ be the specification attached to the \textit{then} composition point
and $s_2$ attached to the \textit{else} composition point, and:

\medskip
\[
  \begin{array}{rcl}
    A(s_1) & = & (Q_1, T_1 \cup R_1, q_{10}, S_1, F_1) \\
    A(s_2) & = & (Q_2, T_2 \cup R_2, q_{20}, S_2, F_2)
  \end{array}
\]
\medskip

For \textit{if-then} specifications composition becomes:

\medskip
\[
  \begin{array}{rcl}
  A(i_E \circ_{then} s_1) & = & (Q, P, q_0, S, F) \\
                        Q & = & Q_1 \cup \{q_0, q_1\} - \{q_{10}\} \\
                        P & = & T'_1 \cup R_1 \cup X \\
                     T'_1 & = & \{(q_0, q, E       \wedge E') \, | \, (q_{10}, q, E') \in T_1\} \\
                        X & = & \{(q_0, q_1, \neg E)\} \\
                        S & = & S_1 \cup \{q_1\} \\
                        F & = & F_1
  \end{array}
\]
\medskip

And for \textit{if-then-else} specifications composition becomes:

\medskip
\[
  \begin{array}{rcl}
  A((i_E \circ_{then} s_1) \circ_{else} s_2) & = & (Q, P, q_0, S, F) \\
                                           Q & = & Q_1 \cup Q_2 \cup \{q_0\} - \{q_{10}, q_{20}\} \\
                                           P & = & T'_1 \cup T'_2 \cup R_1 \cup R_2 \\
                                        T'_1 & = & \{(q_0, q, E       \wedge E') \, | \, (q_{10}, q, E') \in T_1\} \\
                                        T'_2 & = & \{(q_0, q, \neg E \wedge E') \, | \, (q_{20}, q, E') \in T_2\} \\
                                           S & = & S_1 \cup S_2 \\
                                           F & = & F_1 \cup F_2
  \end{array}
\]
\medskip




%================================================
%====== Formal Foundation - Semantics
%================================================

\subsection{Semantics} \label{section-approach-formal-foundation-semantics}

The semantics for the formal specifications described above is quite
straightforward. The automata representations and composition rules preserve
the semantics of the Python code, and together with a description of a system
model we can deduce whether a specification would accept or reject an instance
of such a model.

\subsubsection{Generating a System Model-Specification Product}

The system model in \textit{pythonrv} is quite simple.

\begin{mydef}[\textit{pythonrv} system model]\label{def-system-model}
The \textit{system model} consists of a sequence of \textit{events} of function
calls, $S = (\alpha_0, \alpha_1, \dots, \alpha_n)$. Each event $\alpha_i =
(\lambda_i, \delta_i)$ is a two-part structure: The function called, $\lambda$,
and the associated arguments and state, $\delta$. Since the system model is an
abstraction of an execution of the system (the target program), it is
necessarily finite, but of arbitrary length, and possibly extendible through
continued execution of the program.

The system model can be viewed as a directed acyclic graph (actually, a tree
--- even a list), where all nodes but the last have one edge, with edges
denoting events. The first edge is labeled $\alpha_0$, the second is labeled
$\alpha_1$, etc. See Figure~\ref{figure-deduction-example}~(a) for an example.
A system model $M$ represented as a graph consists of a set tuples
$\{(s_0,s_1,\alpha_0), (s_1,s_2,\alpha_1), \dots, (s_n,s_{n+1},\alpha_n)\}$.
\end{mydef}


In the same way that the system model can be viewed as a graph, so can a
specification automata, see Figure~\ref{figure-deduction-example}~(b). All
graphs have a depth:

\begin{mydef}[graph levels and depth]
The \textit{level} of a node $q$ in a graph rooted at a node $q_0$ is the
number of transitions required to reach the $q$ starting from $q_0$. $q_0$ lies
at level $0$. The \textit{depth} of a graph is its greatest level. The
depth of a graph containing cycles is considered infinite, and nodes lying on a
cycle lie at levels $i$, $i+\text{cycle circumference}$, $i+2\cdot\text{cycle
circumference}$, etc.
\end{mydef}

The \textit{product} of a system model and a specification automata is the main
construct used to reason about the correctness of the specification. An example
can be seen in Figure~\ref{figure-deduction-example}~(c).

\begin{mydef}[product of a system model and a specification]
Let $M$ be a system model represented as a graph, rooted at $s_0$, consisting
of tuples $(s,s',\alpha) \in M$. Let $C$ be a specification automata $(Q, P,
q_0, S, F)$, which is rooted at $q_0$. The \textit{product} $Z = M \times C$ is
created through a recursive procedure:

\medskip
\[
  Z = (L_{Q_Z}(q_0,s_0), L_{P_Z}(q_0,s_0), (q_0,s_0), L_{S_Z}(q_0,s_0), L_{F_Z}(q_0,s_0)
\]

\[
  \begin{array}{rcl}
   L_{Q_Z}(q,s) & = & \cup_{(q',s') \in L_{Q'}(q,s)} L_{Q_Z}(q',s') \cup L_{Q'}(q,s) \cup \{(q,s)\} \\
   L_{P_Z}(q,s) & = & \cup_{(q',s') \in L_{Q'}(q,s)} L_{P_Z}(q',s') \cup L_{P'}(q,s) \\
    L_{Q'}(q,s) & = & \{(q',s') \, | \, (s,s',\alpha) \in M, (q,q',E) \in P\} \\
    L_{P'}(q,s) & = & \{((q,s), (q',s'), \alpha \models E) \, | \, (s,s',\alpha) \in M, (q,q',E) \in P\} \\
                & & \\
   L_{S_Z}(q,s) & = & \{(q',s') \, | \, (q',s') \in L_{Q_Z}(q,s) \wedge q' \in S\} \\
   L_{F_Z}(q,s) & = & \{(q',s') \, | \, (q',s') \in L_{Q_Z}(q,s) \wedge q' \in F\} \\
  \end{array}
\]
\medskip

This essentially says that $Z$ is the level-wise merging of all states in $C$
at level $i$ with all states in $M$ at level $i$, and with the transitions
leading to them. $q_0$ is merged with $s_0$, and all initial transitions in $C$
are merged with the single initial transition in $M$, and so on. We create new
labels for each transition of the form $\alpha \models E$, where $\alpha$ is
the label of the system model transition, and $E$ is the label of the
specification transition. See below for what this means.

A state $(q,s)$ in $L_{Q_Z}(q_0,s_0)$ is a success state if $q \in S$ and a
failure state if $q \in F$.

\end{mydef}

In Figure~\ref{figure-deduction-example} we show an example of a system model,
a specification, and the resulting product of combining them.

\begin{mylemma}[a product $Z$ is a tree]
The product $Z$ will be a tree --- a connected, acyclic graph. This is given by
the procedure above:
\begin{inparaenum}[1\upshape)]
  \item $Z$ will be connected, as each node at level $i > 0$ has an incoming
    transition from a node at level $i-1$, and the root has transitions to all
    nodes at level $1$; and
  \item $Z$ will be acyclic, as edges only go from nodes at level $j$ to nodes
    at level $j+1$, and each node has at most one incoming edge.
  \end{inparaenum}
\end{mylemma}

\begin{mylemma}[a product $Z$ is finite]
Nodes in $Z$ at level $i$ are created from nodes in $C$ and $M$ at level $i$
--- if there are no nodes at level $i$ in $Z$, then are no nodes at level $i$
in either $C$ or $M$. Since the depth of $M$ is by definition finite, there
exists an $k$ such that there are no nodes at levels $\geq k$ in $Z$. In fact,
$k = min(depth(M),depth(C))$.
\end{mylemma}


%%%%%%%%%%%%%%%%%%%%%%%%
%% Deduction example
%%%%%%%%%%%%%%%%%%%%%%%%
\begin{figure}[h!]
	\begin{minipage}{0.9\textwidth}
		\centering
		\begin{tikzpicture}[->,shorten >=1pt,node distance=3cm,on
      grid,auto,scale=.5,semithick,>=stealth]
			\node[state,initial] (s0) {$s_0$};
      \node[state] (s1) [right of=s0] {$s_1$};
      \node[state] (s2) [right of=s1] {$s_2$};
      \node[state] (s3) [right of=s2] {$s_3$};
			\path
				(s0) edge node {$\alpha_0 = (a, \epsilon)$} (s1)
				(s1) edge node {$\alpha_1 = (b, \epsilon)$} (s2)
				(s2) edge node {$\alpha_2$} (s3);
		\end{tikzpicture}
    \subcaption{The system model $M$ consisting of two function calls, to $a$
    and $b$, and an unspecified event $\alpha_2$.}
	\end{minipage}

  \bigskip

	\begin{minipage}{0.90\textwidth}
		\centering
		\begin{tikzpicture}[->,shorten >=1pt,node distance=3cm,on
      grid,auto,scale=.5,semithick,>=stealth]
			\node[state,initial] (q0) {$q_0$};
      \node[state] (q1) [below right of=q0] {$q_1$};
      \node[state] (f) [below left of=q0] {$f$};
      \node[state,accepting] (q2) [below right of=q1] {$q_2$};
			\path
        (q0) edge node {$\lambda = a$} (q1)
        (q0) edge node [above left] {$\lambda \neq a$} (f)
        (q1) edge node {$\lambda = b$} (q2)
        (q1) edge node [above] {$\lambda \neq b$} (f);
		\end{tikzpicture}
    \subcaption{The specification automata $C$. This automata is designed to
      accept system models that consists of one call to $a$ followed by a call to
      $b$. After that, verification is complete and any events are accepted.
      $f$ is a fail state and $q_2$ is a success state.}
	\end{minipage}

  \bigskip

	\begin{minipage}{0.90\textwidth}
		\centering
		\begin{tikzpicture}[->,shorten >=1pt,node distance=3cm,on
      grid,auto,scale=.5,semithick,>=stealth]
			\node[state,initial] (q0) {$q_0,s_0$};
      \node[state] (q1) [below right of=q0] {$q_1,s_1$};
      \node[state] (f1) [below left of=q0] {$f_1,s_1$};
      \node[state,accepting] (q2) [below right of=q1] {$q_2,s_2$};
      \node[state] (f2) [below left of=q1] {$f_2,s_2$};
			\path
        (q0) edge node {$(a, \epsilon) \models \lambda = a$} (q1)
        (q0) edge node [above left] {$(a, \epsilon) \models \lambda \neq a$} (f1)
        (q1) edge node {$(b, \epsilon) \models \lambda = b$} (q2)
        (q1) edge node [above left] {$(b, \epsilon) \models \lambda \neq b$} (f2);
		\end{tikzpicture}
    \subcaption{The product of the model and the specification, $M \times C$.
    $(f_1,s_1)$ and $(f_2,s_2)$ are fail states; $(q_2,s_2)$ is a success state.}
	\end{minipage}

  \caption{Deduction example.}
	\label{figure-deduction-example}
\end{figure}
%%%%%%%%%%%%%%%%%%%%%%%%
%%%%%%%%%%%%%%%%%%%%%%%%
%%%%%%%%%%%%%%%%%%%%%%%%


\subsubsection{Verification}

The transition labels for the product of a specification automata and a system
model is of the form (the second line is a expanded form of the first):

\medskip
\[
  \begin{array}{rcl}
    \alpha = (\lambda, \delta) & \models & \text{expression $E$ over $\alpha (\lambda, \delta)$} \\
  \end{array}
\]
\medskip

The expression $E$ is translated directly from Python expressions in the
specification function. An incomplete sketch of the translation procedure,
shown in Figure~\ref{figure-semantics-translation}, is just syntax replacement.
The resulting specification expression has the same semantics as the Python
expression. In the same way that the Python specification operates on an event,
a specification expression is evaluated to true ($\top$) or false ($\bot$)
using the values of an event $\alpha$.

\begin{figure}[h!]
	\begin{minipage}{0.9\textwidth}
		\centering
    \begin{tabular}{r c l}
      $B(\texttt{True)})$ & = & $\top$ \\
      $B(\texttt{False)})$ & = & $\bot$ \\
      $B(\texttt{event.fn.x.called})$ & = & $\lambda = x$ \\
      $B(\texttt{event.called\_function == x})$ & = & $\lambda = x$ \\
      $B(\texttt{event.fn.x.property})$ & = & $\delta(x.\texttt{property})$ \\
      $B(\texttt{not a)})$ & = & $\neg B(\texttt{a})$ \\
      $B(\texttt{a == b)})$ & = & $B(\texttt{a}) = B(\texttt{b})$ \\
      $B(\texttt{a != b)})$ & = & $B(\texttt{a}) \neq B(\texttt{b})$ \\
      $B(\texttt{a and b)})$ & = & $B(\texttt{a}) \wedge B(\texttt{b})$ \\
      $B(\texttt{a or b)})$ & = & $B(\texttt{a}) \vee B(\texttt{b})$ \\
      $B(\textit{A Python function or property})$ & = & \textit{The return value} \\
    \end{tabular}
  \end{minipage}
  \caption{A sketch of the translation procedure of Python expressions to
    specification expressions. $B$ is the mapping function, replacing the
    Python syntax with mathematics syntax.}
  \label{figure-semantics-translation}
\end{figure}



A product $Z$ of a system model and a specification can be used to determine
whether that specification would accept or reject the system model --- or not
know to do either or, yet.

Informally, the \textit{verification procedure} is to:

\begin{enumerate}
  \item Compute the product $Z$ of the specification $C$ and the system model
    $M$, as described above.

  \item Traverse $Z$, starting at the root node, following all transitions
    which labels evaluate to true.

    \begin{enumerate}
      \item If a fail state is reached, the system model violates the
        specification.
      \item If only success states are reached, the system model satisfies the
        specification.
      \item If some traversal ends in a state that is neither a fail or a
        success state, the system model could, if it were extended, either
        satisfy or violate the specification.
    \end{enumerate}
\end{enumerate}

This can be formally captured with \textit{operational semantics}:


\begin{mydef}[transition function]
Given a product $Z$ of a specification $C$ and a system model $M$, such that
$Z$ is an automata $(Q, P, q_0, S, F)$, the semantics of a transition
$\rightarrow$ between two states $q$ and $q'$, where $q,q' \in Q$ and
\textit{B} is the mapping function from
Figure~\ref{figure-semantics-translation}, is defined as:

\medskip
\[
  \frac{\exists (q, q', E) \in P, \textit{B}(E) = \top}
  {\langle(Q, P, q_0, S, F), x\rangle \rightarrow \langle(Q, P, q_0, S, F), x'\rangle}
\]
\medskip
\end{mydef}

\begin{mydef}[transitive closure of the transition function]
Let $A$ be the automata of $Z$, a product of a system model and a
specification, and let $\xrightarrow*$ be the \textit{transitive closure} of
the transition function, with the semantics that, informally, $\langle A,
q\rangle \xrightarrow* \langle A, q'\rangle$ means that $q'$ can be reached,
starting from $q$, using the transition function $\rightarrow$, traversing zero
or more intermediate states. Formally:

\medskip
\[
  \frac{}
  {\langle A, q\rangle \xrightarrow* \langle A, q\rangle}
  \quad
  \frac{\langle A, q\rangle \rightarrow \langle A, q'\rangle}
  {\langle A, q\rangle \xrightarrow* \langle A, q'\rangle}
  \quad
  \frac{\langle A, q\rangle \xrightarrow* \langle A, q''\rangle, \langle A, q''\rangle \rightarrow \langle A, q'\rangle}
  {\langle A, q\rangle \xrightarrow* \langle A, q'\rangle}
\]
\medskip
\end{mydef}

\begin{mydef}[formal verification]

With this we can determine whether the system model $M$ satisfies, violates or
neither-satisfies-nor-violates the specification $C$:

\begin{enumerate}
  \item If $\exists q' \in F, q_0 \xrightarrow* q'$ then $M$ violates $C$.
    There exists a failure state that is reachable from $q_0$.
  \item If $\forall q' \in Q, q_0 \xrightarrow* q' \Rightarrow q' \in S$ then
    $M$ satisfies $C$. All states reachable from $q_0$ are success states.
  \item If neither condition 1 or condition 2 holds, then $M$ neither satisfies
    nor violates $C$ yet. There exists a state reachable from $q_0$ that is
    neither a success state nor a failure state.
\end{enumerate}

\end{mydef}

Formal \textit{pythonrv} specifications are verified using this procedure,
although in an incremental manner. The verification is done online, so as soon
as a new event is added to the system model, it is combined with the
specification automata.





%================================================
%====== Formal Foundation - Examples
%================================================

\subsection{Examples of Formal Specifications}
\label{section-approach-examples-of-formal-specifications}
\lstset{language=Python,numbers=none}

Here follows two examples of formal \textit{pythonrv} specifications, both in
their specification function form, written in Python, shown in
Figure~\ref{figure-formal-syntax-example-1} and
Figure~\ref{figure-formal-syntax-example-2}.

The first example, Figure~\ref{figure-formal-syntax-example-1}, shows a simple
specification that makes sure that the first argument to the function
\texttt{fib} in the module \texttt{fibmodule} is always positive (\texttt{fib}
is relabeled \texttt{func} in the specification). The main difference with the
similar informal specification in Figure~\ref{figure-syntax-example-1} is that
here we have to explicitly create a loop by composing with a \textit{next}
specification, pointing back to \texttt{spec}. This is required so that not
only the first event is verified.

In Figure~\ref{figure-formal-syntax-example-1-product} we combine this
specification with a system model of two calls to \texttt{fib}, the first with
the first argument as $0$, the second with the first argument as $-1$. As we
traverse this graph, we can see that we cannot reach state $(f, s_1)$, because
the label on the transition to it evaluates to $\bot$:

\medskip
\[
  \begin{array}{rcl}
    (\lambda_0, \delta_0) & \models & \neg (\delta(\texttt{func.inputs$[0]$}) > 0) \\
    (\textit{func}, \textit{func.inputs}[0] = 0) & \models & \neg (\delta(\texttt{func.inputs$[0]$}) > 0) \\
  \end{array}
\]
\medskip

The first event of the system model makes it possible to reach state
$(q_1,s_1)$, which is an accepting state. Thus we continue, and reach
$(q_0,s_0)'$, $(q_0,s_0)''$ and $(f,s_1)'$, the last which is a fail state. The
system model does not satisfy the specification.

The example in Figure~\ref{figure-formal-syntax-example-2} is more complex, as
can be seen by the sprawling automata in
Figure~\ref{figure-formal-syntax-example-2-automata}. The specification accepts
only interleaving calls to two functions, starting with \texttt{mymodule.foo},
then to \texttt{mymodule.bar}. This is represented by the loop between the
$q_0$ and $r_0$ states in the automata. Because the labels for the transitions
between them are $\top$, i.e.\ always true, we are guaranteed that on every
event we either transition into a state (in the system model-specification
product) based containing $q_0$ or $q_1$, and verification will never be
finished. The specification also asserts that if \texttt{foo} is called with
$0$ as the first argument, the second argument must also be $0$.  This is just
to demonstrate how the \textit{if-then} construct works.


%%%%%%%%%%%%%%%%%%%%%%%%
%% formal syntax examples
%%%%%%%%%%%%%%%%%%%%%%%%

%%%%%%%%%%%%%%%%%%%%%%%%
%% first
%%%%%%%%%%%%%%%%%%%%%%%%
\begin{figure}[h!]
	\begin{center}
	\begin{minipage}{0.9\textwidth}
	\lstinputlisting{figures/formal_syntax_example_1.py}
    \subcaption{A Python representation of the specification.}
	\end{minipage}
	\end{center}

  \bigskip

	\begin{minipage}{0.9\textwidth}
		\centering
    \[
      \begin{array}{rcl}
        A(\text{spec}) & = & (Q, P, q_0, S, F) \\
                     Q & = & \{q_0, q_1, f\} \\
                     P & = & \{(q_0, q_1, \delta(\texttt{func.inputs$[0]$}) > 0),
      (q_0, f, \neg(\delta(\texttt{func.inputs$[0]$}) > 0)), \\
                       &   & \; (q_0, q_0, \top)\} \\
                     S & = & \{q_1\} \\
                     F & = & \{f\}
      \end{array}
    \]
    \subcaption{An automata representation of the specification.}
	\end{minipage}

  \bigskip

	\begin{minipage}{0.9\textwidth}
		\centering
		\begin{tikzpicture}[->,shorten >=1pt,node distance=3.5cm,on
      grid,auto,scale=.5,semithick,>=stealth]
			\node[state,initial] (q0) {$q_0$};
			\node[state,accepting] (q1) [below right of=q0] {$q_1$};
      \node[state] (f) [below left of=q0] {\textit{f}};
			\path
        (q0) edge [loop above] node {$\top$} (q0)
        (q0) edge node {$\delta(\texttt{func.inputs$[0]$}) > 0$} (q1)
        (q0) edge node [above left] {$\neg(\delta(\texttt{func.inputs$[0]$}) > 0)$} (f);
		\end{tikzpicture}
    \subcaption{A graphical representation of the automata.}
	\end{minipage}
  \caption{A formal \textit{pythonrv} specification, similar to the informal
    specification shown in Figure~\ref{figure-syntax-example-1}.}
	\label{figure-formal-syntax-example-1}
\end{figure}

\begin{figure}[h!]
	\begin{minipage}{0.9\textwidth}
		\centering
		\begin{tikzpicture}[->,shorten >=1pt,node distance=4cm,on
      grid,auto,scale=.5,semithick,>=stealth]
			\node[state,initial] (s0) {$s_0$};
      \node[state] (s1) [above right of=s0] {$s_1$};
      \node[state] (s2) [below right of=s1] {$s_2$};
			\path
      (s0) edge node {$(\lambda_0 = \textit{func}, \delta_0 = \textit{func.inputs}[0] = 0)$} (s1)
      (s1) edge node {$(\lambda_1 = \textit{func}, \delta_1 = \textit{func.inputs}[0] = -1)$} (s2);
		\end{tikzpicture}
    \subcaption{A system model $M$ consisting of two function calls to
      \texttt{func} the first with the first argument as $0$, the other with
      it as $-1$.}
	\end{minipage}

  \bigskip

	\begin{minipage}{0.95\textwidth}
		\centering
		\begin{tikzpicture}[->,shorten >=1pt,node distance=3.5cm,on
      grid,auto,scale=.5,semithick,>=stealth]
			\node[state,initial] (q0) {$(q_0,s_0)$};
      \node[state] (q0') [below of=q0] {$(q_0,s_0)'$};
      \node[state,accepting] (q1) [right of=q0'] {$(q_1,s_1)$};
      \node[state] (f1) [left of=q0'] {$(f,s_1)$};
      \node[state] (q0'2) [below of=q0'] {$(q_0,s_0)''$};
      \node[state,accepting] (q1') [right of=q0'2] {$(q_1,s_1)'$};
      \node[state] (f1') [left of=q0'2] {$(f,s_1)'$};
			\path
        (q0) edge node {$(\lambda_0, \delta_0) \models \delta(\texttt{func.inputs$[0]$}) > 0$} (q1)
        (q0) edge node [above left] {$(\lambda_0, \delta_0) \models \neg (\delta(\texttt{func.inputs$[0]$}) > 0)$} (f1)
        (q0) edge node [right] {$\top$} (q0')
        (q0') edge node {$(\lambda_1, \delta_1) \models \delta(\texttt{func.inputs$[0]$}) > 0$} (q1')
        (q0') edge node [above left] {$(\lambda_1, \delta_1) \models \neg (\delta(\texttt{func.inputs$[0]$}) > 0)$} (f1')
        (q0') edge node [right] {$\top$} (q0'2);
		\end{tikzpicture}
    \subcaption{The product of the model above and the specification.}
	\end{minipage}

  \caption{Example of creating the product of the specification from
    Figure~\ref{figure-formal-syntax-example-1} with a system model.}
	\label{figure-formal-syntax-example-1-product}
\end{figure}

%%%%%%%%%%%%%%%%%%%%%%%%
%% second
%%%%%%%%%%%%%%%%%%%%%%%%

\begin{figure}[h!]
	\begin{center}
	\begin{minipage}{0.9\textwidth}
	\lstinputlisting{figures/formal_syntax_example_2.py}
	\end{minipage}
	\end{center}

  \caption{A more complex formal \textit{pythonrv} specification function. The automata for
    this specification function is shown in
    Figure~\ref{figure-formal-syntax-example-2-automata}.}
	\label{figure-formal-syntax-example-2}
\end{figure}

\begin{figure}[h!]
	\begin{minipage}{0.9\textwidth}
		\centering
    \[
      \begin{array}{rcl}
        A(\text{spec}) & = & (Q, P_1 \cup P_2 \cup P_3 \cup P_4, q_0, S, F) \\
                     Q & = & \{q_0, q_1, f_1, q_2, q_3, f_2, r_0, r_1, f_r\} \\
                   P_1 & = & \{(q_0, q_1, \lambda = \texttt{foo}), (q_0, f_1, \neg(\lambda = \texttt{foo}))\} \\
                   P_2 & = & \{(q_0, q_2, G \wedge X), (q_0, f_2, G \wedge \neg X), (q_0, q_3, \top)\} \\
                   P_3 & = & \{(q_0, r_0, \top), (r_0, q_0, \top)\} \\
                   P_4 & = & \{(r_0, r_1, \lambda = \texttt{bar}), (r_0, f_r, \neg(\lambda = \texttt{bar}))\} \\
                     S & = & \{q_1, q_2, q_3, r_1\} \\
                     F & = & \{f_1, f_2, f_r\} \\
                     G & = & \delta(\texttt{foo.inputs$[0]$}) = 0 \\
                     X & = & \delta(\texttt{foo.inputs$[1]$}) = 0
      \end{array}
    \]
	\end{minipage}

  \bigskip

	\begin{minipage}{0.9\textwidth}
		\centering
		\begin{tikzpicture}[->,shorten >=1pt,node distance=3.5cm,on
      grid,auto,scale=.5,semithick,>=stealth]

      % first spec
			\node[state,initial above] (q0) {$q_0$};
			\node[state,accepting] (q1) [left of=q0] {$q_1$};
      \node[state] (f1) [above left of=q0] {\textit{f$_1$}};
      % if
			\node[state,accepting] (q2) [below right of=q0] {$q_2$};
      \node[state] (f2) [below of=q0] {\textit{f$_2$}};
			\node[state,accepting] (q3) [below left of=q0] {$q_3$};

      % second spec
      \node[state] (r0) [right of=q0] {$r_0$};
			\node[state,accepting] (r1) [right of=r0] {$r_1$};
      \node[state] (fr) [above right of=r0] {\textit{f$_r$}};
			\path
        % first spec
        % assert
        (q0) edge node [above] {$\lambda = \text{foo}$} (q1)
        (q0) edge node [above right] {$\neg(\lambda = \text{foo})$} (f1)
        % if
        (q0) edge node [above right] {$G \wedge X$} (q2)
        (q0) edge node [right] {$G \wedge \neg X$} (f2)
        (q0) edge node [above left] {$\neg G$} (q3)
        % next
        (q0) edge [bend left=20] node [above] {$\top$} (r0)

        % second spec
        % assert
        (r0) edge node {$\lambda = \text{bar}$} (r1)
        (r0) edge node [above left] {$\neg(\lambda = \text{bar})$} (fr)
        % next
        (r0) edge [bend left=20] node [above] {$\top$} (q0);
		\end{tikzpicture}
	\end{minipage}
  \caption{A direct translation of the formal \textit{pythonrv} specification
    function in Figure~\ref{figure-formal-syntax-example-2} to an automata.
    $q_0$ is the root for the \texttt{spec} specification, and $r_0$ is the
    root of the \texttt{spec\_bar\_called} specification. $q_1$ and $f_1$
    relates to the assertion in \texttt{spec}, and $q_2$, $q_3$ and $f_2$ to
    the if statement. $r_1$ and $f_r$ relates to the assertion in
    \texttt{spec\_bar\_called}.}
	\label{figure-formal-syntax-example-2-automata}
\end{figure}
%%%%%%%%%%%%%%%%%%%%%%%%
%%%%%%%%%%%%%%%%%%%%%%%%
%%%%%%%%%%%%%%%%%%%%%%%%

