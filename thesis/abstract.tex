%================================================
%====== The Abstracts
%================================================

\begin{abstract}

% introduction & problem
Computer software is growing ever more complex, and more sophisticated tools
are required to make sure the software operates in a correct way --- i.e.\
according to its specification. Traditional approaches to program verification
have much to give, but they also have their disadvantages. While formal methods
can give useful mathematical proofs about the correctness of programs, they
suffer from complexity and are difficult to use. Often they work only on a
constructed system model, not the actual program. Testing, on the other hand,
has a simple syntax and great tool support, and it is in widespread use. But it
is informal and incomplete, only testing the specific test cases that the
test-writers can come up with.

A relatively new approach called \textit{runtime verification} is an attempt
for a lightweight alternative. It verifies a program's actual
\textit{execution} against its specification, possibly while the program is
running.

% abstract solution
This work investigates how testing, and specifically unit testing, can be
combined with runtime verification. It shows how the syntax of unit tests,
written in the target program's programming language, can be used to inspire
the syntax for specifications for runtime verification. Both informal and
formal specifications are described and supported.

% deliverables
A proof-of-concept framework for Python called \textit{pythonrv} is implemented
and described, and it is tested on a real-life application. A formal foundation
is constructed for specifications written in a subset of Python, enabling
formal verification. Informal specifications are also supported, with the full
power of Python as specification language.

% results & conclusions
The result shows that the proof-of-concept framework allow for effective use of
runtime verification. It is easy to integrate into existing programs, and the
informal specification syntax is relatively simple. It also shows that formal
specifications can be written in Python, but in a more unwieldy syntax and
structure than the informal one. Many interesting properties can be verified
using it that ordinary tests would have trouble with.

% potential
The recommendation for future work lies in improving the specification syntax,
using unit testing concepts such as expectations, and on working to make the
formal specification syntax more like that of its informal sibling.

\end{abstract}
\clearpage


\begin{foreignabstract}{swedish}

Mjukvarusystem växer sig allt mer komplexa, och mer sofistikerade verktyg
krävs för att säkerställa att system fungerar korrekt --- att de opererar
enligt sina specifikationer. Traditionella tillvägagångssätt för
programverifiering tillför mycket, men de har också sina nackdelar. Formella
metoder kan ge användbara matematiska bevis om korrektheten av program, men de
är komplexa och är svåra att använda. Ofta opererar de bara på en
konstruerad systemmodell, inte det faktiska programmet. Testning, som metod,
har å andra sidan en enkel syntax och bra verktygsstöd, och används i stor
utsträckning. Men testning är informell och ofullständig, och testar bara de
specifika testfall som testkrivarna kan komma på.

En relativt ny metod, kallad \textit{runtime-verifiering}, är tänkt som ett
lättviktigt alternativ som verifierar ett programs faktiska \textit{exekvering}
mot dess specifikation, eventuellt även medan programmet kör.

Avsikten här har varit att undersöka hur testning, och specifikt
enhetstestning, kan kombineras med runtime-verifiering. Detta genom att visa
hur syntaxen för enhetstester, skrivna i programmets programmeringsspråk, kan
användas som inspiration för specifikationer för runtime-verifiering.

En proof-of-concept-implementation för Python kallad \textit{pythonrv}
implementeras och beskrivs, och testas på en verklig applikation. En formell
grund framställs för specifikationer skrivna i en delmängd av Python, vilket
möjliggör formell verifiering. Informella specifikationer stöds också, med hela
kraften av Python som specifikationsspråk.

Resultaten visar att proof-of-concept-implemenationen möjliggör en effektiv
användning av runtime-verifiering. Den är enkel att integrera i existerande
program, och den informella specifikationssyntaxen är förhållandevis enkel. Den
visar också att formella specifikationer kan skrivas i Python, men i en mer
omständigt syntax och struktur än den informella. Många intressanta egenskaper,
som vanliga tester skulle ha problem med, kan verifieras med denna metod.

Rekommendationen för framtida arbete är att förbättra specifikationssyntaxen,
genom att använda koncept från enhetstestning såsom förväntningar, och genom
att göra den formella specifikationssyntaxen mer som den av sin informella
kusin.

\end{foreignabstract}
