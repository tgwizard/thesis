%================================================
%====== Chapter 6, Conclusions
%================================================

\pagestyle{newchap}
\chapter{Conclusions} \label{chapter-conclusions}

In this report, and with the proof-of-concept implementation \textit{pythonrv},
we have shown that it is possible to write specifications in the programming
language Python and in a manner more similar to unit testing. Our approach
utilizes the dynamic nature of Python to do instrumentation. Specifically, we
use decorators to mark what functions should be monitored for a specification,
and reflection to deduce the properties of the monitorees. We also use the fact
that each function has a parent container, so it can be rewritten during
runtime with the monitoring code.

However, our approach would work in other programming languages as well if they
support hot-swapping of code --- replacing objects during runtime, injecting
wrapping code around the functions to monitor.

We have also implemented support in \textit{pythonrv} for formal verification.
The formal specifications are written in a subset of Python, which is given a
formal semantics, see Section~\ref{section-approach-formal-foundation}.

A few reservations should be mentioned, however.

The specifications' explicit dealing with time and the actual
execution flow leads to some inherent divergences from ordinary unit testing
styles. This is best exemplified by the \texttt{event.next} method described in
Chapter~\ref{chapter-approach}.

Giving the specifications a formal foundation, and doing formal verification
with them, has been difficult, due to the fact that the chosen programming
language, Python, does not have a formal semantics defined. We define the
formal semantics for a subset of Python, which makes the math easier, but the
resulting syntax less interesting --- it requires a lot of boilerplate
for even the simplest of specifications.

\section{Other Approaches}

In Chapter~\ref{chapter-intro-to-rv} we describe approaches that others have
taken to runtime verification. The formal specification languages used are
mostly LTL or variations thereof, and either written as comments or special
attributes in code, or as separate specification files. See e.g.\ Bauer et
al.\ \cite{bauer06monitoring}, Bodden \cite{bodden05efficientrv}, Kähkönen et
al.\ \cite{kahkonen09lime}, Java PathExplorer by Havelund et al.\
\cite{havelund04jpax}, Temporal Rover by Drusinsky
\cite{drusinsky00temporalrover}; these all use a language for specifications
separate from the target program's programming language.

The informal approaches, like some implementations of Design by Contract, and
simple assertions, often use the target program's programming language. The
contracts and assertions are often heavily attached to the code they verify,
e.g.\ in \cite{bartetzko01jass,meyer92applyingdbc}.

\textsc{LogScope} \cite{barringer09tutorial} is one of the few runtime
verification approaches that are written in Python. It is very different from
\textit{pythonrv} --- its specifications are written in a temporal logic based
on \textsc{RuleR}, and builds its system model from log files, working
completely offline.

\section{Future Work}

The testing tool called expectations, as described in
Section~\ref{section-expectations}, could fit quite well with the
\textit{pythonrv} style of writing specifications. This could allow for a
cleaner and more succinct way to describe temporal properties, perhaps
obviating the need to use the \texttt{event.next} method in many cases.

Especially the formal specification could do with more work on the
specification language. Can this language become more like that of the informal
specifications?

The performance of the implementation has not been measured or considered in
much detail. Benchmark tests for \textit{pythonrv} would be interesting, as
would attempts to introduce it as a correctness verification approach for more
programs.

Offline verification, discussed in Section~\ref{section-rv} and
Section~\ref{section-approach-verification} would be interesting. A simple way
to turn verification on or off, or to switch between online and offline, would
be nice. For instance, when a bug has been found, RV could be turned on for
further verification and help in finding the erroneous code.

If the verification parts of \textit{pythonrv} is unwanted, it could be used as
a simple framework for aspect-oriented programming. Self-healing and
self-adaptation would be a very interesting use. Extracting logging
functionality into separate modules, written as \textit{pythonrv}
specifications, could make the separation of concerns in the program clearer.

It would be interesting to investigate how the approach we have used in this
report would work in languages beside Python. Both Java and Ruby, for instance,
support hot-swapping of code, but an implementation of \textit{pythonrv} in
those languages would look very different than the original, due to the
underlying workings of those languages.


\section{Final Words}

The trend of software systems in general seems to be toward larger and more
complex entities. This makes the automated verification of program
correctness, formal or not, ever more important and an essential part of
software development. Runtime verification could have a place there, if it
becomes more popular and simpler to integrate and use in ordinary software.

The implementation described in this report, \textit{pythonrv}, is publicly
available on the web\footnote{\texttt{https://github.com/tgwizard/pythonrv}} as
free, open source software. People are welcome to try it, incorporate it into
their programs, and extend it, as they see fit. With enough interest,
\textit{pythonrv} might develop into a mature framework for runtime
verification.

