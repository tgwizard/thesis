\documentclass[a4paper,11pt]{kth-mag}
\usepackage[T1]{fontenc}
\usepackage{textcomp}
\usepackage{lmodern}
\usepackage[latin1]{inputenc}
\usepackage[swedish,english]{babel}
\usepackage{modifications}
\usepackage{draftwatermark}
%\usepackage[firstpage]{draftwatermark}

\SetWatermarkLightness{0.9}
\SetWatermarkScale{1.1}

\title{Test-oriented runtime verification}

\subtitle{Using a test-like specification syntax for runtime verification}
          
\foreigntitle{"Test-orienterad runtime-verifiering"}
              
\author{Adam Renberg}
\date{May 2012}
\blurb{Master's Thesis at CSC\\Supervisor Valtech: Erland Ranvinge\\Supervisor CSC: Narges Khakpour\\Examiner: Tjohej}
\trita{TRITA xxx yyyy-nn}

\begin{document}
\frontmatter
\pagestyle{empty}
\removepagenumbers
\maketitle

\selectlanguage{english}
\begin{abstract}
  This is a skeleton for KTH theses. More documentation
  regarding the KTH thesis class file can be found in
  the package documentation.

Lorem ipsum dolor sit amet, consectetuer adipiscing elit. Mauris
purus. Fusce tempor. Nulla facilisi. Sed at turpis. Phasellus eu
ipsum. Nam porttitor laoreet nulla. Phasellus massa massa, auctor
rutrum, vehicula ut, porttitor a, massa. Pellentesque fringilla. Duis
nibh risus, venenatis ac, tempor sed, vestibulum at, tellus. Class
aptent taciti sociosqu ad litora torquent per conubia nostra, per
inceptos hymenaeos.

\bigskip\noindent
Keywords: 
\end{abstract}
\clearpage

\begin{foreignabstract}{swedish}
  Denna fil ger ett avhandlingsskelett.
  Mer information om \LaTeX-mallen finns i
  dokumentationen till paketet.

Lorem ipsum dolor sit amet, consectetuer adipiscing elit. Mauris
purus. Fusce tempor. Nulla facilisi. Sed at turpis. Phasellus eu
ipsum. Nam porttitor laoreet nulla. Phasellus massa massa, auctor
rutrum, vehicula ut, porttitor a, massa. Pellentesque fringilla. Duis
nibh risus, venenatis ac, tempor sed, vestibulum at, tellus. Class
aptent taciti sociosqu ad litora torquent per conubia nostra, per
inceptos hymenaeos.

\bigskip\noindent
Keywords (S\"okord? Nyckelord?): 
\end{foreignabstract}
\clearpage


\pagestyle{newchap}
\chapter*{Preface}

This is a master thesis / exjobb in Computer Science at the Royal Institute of Technology (KTH), Stocholm. The work was done at Valtech Sweden, an IT Consultancy. It was supervised by Erland Ranvinge (Valtech) and Dr. (TODO: check) Narges Khakpour (CSC KTH).

Thanks to people.
\clearpage

\pagestyle{newchap}
\tableofcontents*
\mainmatter


\pagestyle{newchap}
\chapter{Introduction}

This is the introduction.

Purpose: Test-like syntax of runtime verification specifications.

What will this report discuss? What problems? Why is this interesting?

What will this report \textbf{not} discuss?

Perhaps: Discuss the sectioning of this report.

\pagestyle{newchap}
\chapter{Background}

Previous work, in correctness and RV.

\section{Proving Correctness}

\subsection{Formal Verification}

Best result. Tedious. Often impossible.

\subsection{Model Checking}

Nice, simpler than formal verification. Can yield impossibly large state spaces.

\subsection{Testing}

Not formal - doesn't prove anything except for the specified test cases.

Manual. Automatic test-generation?

\section{Runtime Verification}

The idea: Lightweight formal verification. Execution trace. Speed? Monitoring.

\subsection{Writing Specifications}

LTL. TLTL. EAGLE?

\subsection{Transforming Specifications and Instrumenting Code}

B\"uchi Automatons.
AspectJ.

\subsection{Online v. Offline}

\pagestyle{newchap}
\chapter{Test-Oriented Runtime Verification}

What have I done, and why (again)?

\section{Testing Frameworks}

How do they work? What are their syntaxes?

\section{Comparing Testing Frameworks, Languages and Environments}

Why this testing framework as starting point? Why this language?

\section{pyrv}

\subsection{General}

\subsection{Syntax?}

\subsection{Correctness}


\section{Conclusions}

It is all awwwesomee!

\pagestyle{newchap}
\chapter{Discussion}

What do we see in the future? How can this be extended, continued?

Results (un)expected? Larger context.

Some speculation? Recommendations?

\appendix
\addappheadtotoc
\chapter{RDF}\label{appA}

\begin{figure}[ht]
\begin{center}
And here is a figure
\caption{\small{Several statements describing the same resource.}}\label{RDF_4}
\end{center}
\end{figure}

that we refer to here: \ref{RDF_4}
\end{document}
